\documentclass{article}
\usepackage[left=3cm,right=3cm,top=3cm,bottom=3cm]{geometry}

\usepackage[utf8]{inputenc}
\usepackage[english]{babel}
\usepackage[OT1]{fontenc}
\usepackage{hyperref}
\usepackage[inference]{semantic}
\usepackage{graphicx}
\usepackage{amsfonts,amsmath}
\usepackage{cleveref}
\usepackage{scalerel}
\usepackage{multicol}
\usepackage{xspace}
\usepackage{amsmath}
\usepackage{amssymb}
\usepackage{stmaryrd}
\usepackage{proof}
\usepackage{natbib}
\usepackage{booktabs}

\usepackage{mathtools}
\usepackage{mathpartir}
\usepackage{colonequals}
\usepackage{xspace}
% Commands for correct abbreviations


\usepackage{graphicx}
\usepackage{bussproofs}
\usepackage{tikz}
\usetikzlibrary{shapes,arrows,chains}

\EnableBpAbbreviations

\usepackage{tikz}
\usepackage{tikz-cd}
\usetikzlibrary{calc}
\usetikzlibrary{arrows.meta}
\def\checkmark{\tikz\fill[scale=0.4](0,.35) -- (.25,0) -- (1,.7) -- (.25,.15) -- cycle;}

\usepackage{dsfont}

\newcommand{\lto}{\Rightarrow}
\newcommand{\TT}{\mathcal{T}}
\newcommand{\MM}{\mathcal{M}}
\newcommand{\KK}{\mathcal{K}}
\newcommand{\UU}{\mathcal U}

\renewcommand{\to}{\rightarrow}
\newcommand{\SN}{\textnormal{SN}}
\newcommand{\RED}{\textnormal{RED}}
\newcommand{\NE}{\textnormal{NE}}

\newtheorem{exercise}{Exercise}

\begin{document}

\title{Simply Typed Lambda Calculus}
\author{Dominik Kirst}

\date{LMFI Proof Theory Course\\November 27th, 2025}

\maketitle

\begin{exercise}[Typing Derivations]
	Derive types for the following terms, where possible:
	\begin{multicols}{2}
		\begin{enumerate}
			\item
			$x$
			\item
			$\lambda x.xy$
			\item
			$\lambda xy.x(yxx)$
			\item
			$x(\lambda y.x)$
		\end{enumerate}
	\end{multicols}
	\noindent
	For some of these terms you may have to use a non-empty context.
\end{exercise}

\begin{exercise}[Structural Properties]
	Establish the following structural properties:
	\begin{enumerate}
		\item 
		Strengthening, i.e.~if $\Gamma,(x:\sigma)\vdash t:\tau$ and $x\not\in FV(t)$ then $\Gamma\vdash t:\tau$.
		\item
		Type uniqueness, i.e.~if $\vdash t:\tau$ and $\vdash t:\tau'$ then $\tau=\tau'$.
	\end{enumerate}
	\noindent
	Note that for the latter it is crucial to have abstractions $\lambda x^\sigma.t$ with type annotations.
\end{exercise}

\begin{exercise}[Weak Normalisation]
	Compute the size $o(t)$ for the typeable terms from Exercise 1.
	Convince yourself that for every well-typed term $t$ that is not yet normal it is possible to find a suitable reduction step $t\longrightarrow_\beta t'$ with $o(t)<o(t')$.
\end{exercise}

\begin{exercise}[Products]
	Extend the following two properties of STLC to product types:
	\begin{enumerate}
		\item
		Subject reduction, i.e.~if $\Gamma\vdash t:\sigma $ and $t\longrightarrow_\beta t'$ then $\Gamma\vdash t':\sigma $.
		\item
		Strong normalisation, i.e.~if $\Gamma\vdash t:\sigma $ then $t\in \SN$.
	\end{enumerate}
\end{exercise}

\begin{exercise}[Curry-Howard Isomorphism]
	Give proof terms for the following propositions:
	\begin{multicols}{2}
		\begin{enumerate}
			\item
			$(A\land B\lto C)\lto A\lto B\lto C$
			\item
			$(A\lto B\lto C)\lto A\land B\lto C$
			\item
			$(A\lto C)\lto (B\lto D)\lto A\land B\lto C\land D$
			\item
			$A\land (B\land C)\lto (A\land B)\land C$
		\end{enumerate}
	\end{multicols}
\end{exercise}

\end{document}
