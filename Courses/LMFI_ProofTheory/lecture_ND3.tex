\documentclass[xcolor=dvipsnames,aspectratio=169,handout]{beamer}
\usecolortheme{orchid}
\setbeamertemplate{footline}[frame number]


\usepackage[utf8]{inputenc}
\usepackage[english]{babel}
\usepackage[OT1]{fontenc}
\usepackage{hyperref}
\usepackage[inference]{semantic}
\usepackage{graphicx}
\usepackage{amsfonts,amsmath}
\usepackage{cleveref}
\usepackage{scalerel}
\usepackage{multicol}
\usepackage{xspace}
\usepackage{amsmath}
\usepackage{amssymb}
\usepackage{stmaryrd}
\usepackage{proof}
\usepackage{natbib}
\usepackage{booktabs}

\usepackage{mathtools}
\usepackage{mathpartir}
\usepackage{colonequals}
\usepackage{xspace}
% Commands for correct abbreviations


\usepackage{graphicx}
\usepackage{bussproofs}
\usepackage{tikz}
\usetikzlibrary{shapes,arrows,chains}

\EnableBpAbbreviations

\usepackage{tikz}
\usepackage{tikz-cd}
\usetikzlibrary{calc}
\usetikzlibrary{arrows.meta}
\def\checkmark{\tikz\fill[scale=0.4](0,.35) -- (.25,0) -- (1,.7) -- (.25,.15) -- cycle;}

\renewcommand{\to}{\Rightarrow}
\newcommand{\TT}{\mathcal{T}}
\newcommand{\MM}{\mathcal{M}}
\newcommand{\KK}{\mathcal{K}}
\newcommand{\UU}{\mathcal U}
\renewcommand{\AA}{\mathcal A}

\usepackage{dsfont}

\title[Natural Deduction 3]{Natural Deduction 3}
\author[Dominik Kirst]{Dominik Kirst}

\date[October 28th, 2025]{LMFI Proof Theory Course\\October 28th, 2025}


% BOXES
\usepackage[many]{tcolorbox}
\newtcolorbox{emptybox}[1][]{
	beamer,
	width=(0.7\textwidth),
	% enlarge left by=-3pt,
	titlerule=0mm,
	colframe=white,
	coltitle=black,
	bottom=6pt, 
	top=-12pt,
	left=6pt,  
	right=6pt,
	notitle, 
	adjusted title={},
	outer arc=.5mm,
	arc=.5mm,
	no shadow,
	fuzzy shadow={1mm}{-1mm}{-1.2mm}{.7mm}{black!20},
	interior titled code={}
}

\newtcolorbox{widerbox}[1][]{
	beamer,
	width=(0.8\textwidth),
	% enlarge left by=-3pt,
	titlerule=0mm,
	colframe=white,
	coltitle=black,
	bottom=6pt, 
	top=-12pt,
	left=6pt,  
	right=6pt,
	notitle, 
	adjusted title={},
	outer arc=.5mm,
	arc=.5mm,
	no shadow,
	fuzzy shadow={1mm}{-1mm}{-1.2mm}{.7mm}{black!20},
	interior titled code={}
}

\newtcolorbox{widebox}[1][]{
	beamer,
	width=(\textwidth),
	% enlarge left by=-3pt,
	titlerule=0mm,
	colframe=white,
	coltitle=black,
	bottom=6pt, 
	top=-12pt,
	left=6pt,  
	right=6pt,
	notitle, 
	adjusted title={},
	outer arc=.5mm,
	arc=.5mm,
	no shadow,
	fuzzy shadow={1mm}{-1mm}{-1.2mm}{.7mm}{black!20},
	interior titled code={}
}


% TIMELINE
\usepackage{tikz}
\usepackage{graphicx}
\usepackage{stackengine}
\usetikzlibrary{trees, shapes, calc, positioning, shadows}

\definecolor{darkcerulean}{rgb}{0.063661, 0.257392, 0.477463}
\definecolor{scooter}{rgb}{0.161162, 0.775760, 0.885416}


% #1 – box name
% #2 – left color
% #3 – right color
% #4 – Title (may contain #1)
\newcommand{\newlargebox}[4]{
	\newtcolorbox{#1}[1][]{
		beamer,
		width=\textwidth+7pt,
		enlarge left by=-3pt,
		titlerule=3mm,
		colframe=white,
		coltitle={#2},
		bottom=2pt, 
		top=-4pt,
		left=6pt,  
		toptitle=2pt,
		bottomtitle=-2pt,
		fonttitle=\bfseries\large,
		adjusted title={#4},
		outer arc=.5mm,
		arc=.5mm,
		no shadow,
		fuzzy shadow={1mm}{-1mm}{-1.2mm}{.7mm}{black!20},
		interior titled code={
			\path [left color = {#2}, right color = {#3}]
			(title.south west) + (8pt, 0) rectangle ++(\textwidth-1pt, 0.02);
		}
	}
}


\newlargebox{defaultbox}{darkcerulean}{scooter}{#1} 

\usepackage{fancyvrb}
\usepackage{xcolor}


\begin{document}

%\beamerdefaultoverlayspecification{<+->}
\definecolor{red}{RGB}{204,0,0}
\definecolor{yellow}{RGB}{228,242,31}
\definecolor{green}{RGB}{0,204,0}

\newcommand\refs[1]{%
	\begin{textblock*}{8cm}(0.3cm,9.2cm)%
		\scriptsize {\color{gray}#1}
	\end{textblock*}
}

\begin{frame}
	\maketitle
\end{frame}


\begin{frame}{Recap}
	Last week, we studied semantics of ND and completeness theorems:
	\begin{itemize}
		\vspace{0.3cm}
		\item
		Henkin's completeness proof
		\vspace{0.3cm}
		\item
		Constructive refinement of Henkin's completeness proof
		\vspace{0.3cm}
		\item
		Kripke semantics for intuitionistic ND
		\vspace{0.3cm}
		\item
		Kripke completeness as semantic cut-elimination
	\end{itemize}
\end{frame}

\begin{frame}{Plan for Today}
	\begin{itemize}
		\item 
		Algebraic semantics:
		\begin{itemize}
			\item
			Complete Heyting and Boolean algebras
			\item
			Completeness proofs
		\end{itemize}
		\vspace{0.3cm}
		\item
		Arithmetical systems:
		\begin{itemize}
			\item
			Heyting arithmetic and Peano arithmetic
			\item
			Some meta-theoretical properties
		\end{itemize}
		\vspace{0.3cm}
		\item
		Friedman's A-translation:
		\begin{itemize}
			\item
			Syntactic definition and correctness
			\item
			Applications
		\end{itemize}
	\end{itemize}
	
	\pause
	\vspace{0.5cm}
	First part still based on this paper:\\ \scriptsize \url{https://www.ps.uni-saarland.de/Publications/documents/ForsterEtAl_2021_Completeness.pdf}
\end{frame}


\begin{frame}{Heyting and Boolean Algebras}
	Previous semantics mostly interpreted logical connectives in the meta-theory:
	$$\MM\vDash A\land B \hspace{3em}:=\hspace{3em} \MM\vDash A \text{ ``and'' } \MM\vDash B$$
	\pause
	
	This can be generalised to arbitrary algebraic structures capturing the same behaviour:
	\pause
	\begin{definition}
		We call $\AA=(\Omega,\le,0,1,\sqcap,\sqcup,\sqsupset)$ a \textbf{Heyting algebra} if:
		\begin{itemize}
			\item
			$\le$ is a preorder on $\Omega$ with $0$ being minimal and $1$ being maximal
			\item
			$\sqcap$ forms greatest lower bounds / meets: $z\le x\sqcap y$ if and only if $z\le x$ and $z\le y$
			\item
			$\sqcup$ forms least upper bounds / joins: $x\sqcup y\le z$ if and only if $x\le z$ and $y\le z$
			\item
			$\sqsupset$ forms Heyting implication: $z\le x\sqsupset y$ if and only if $z\sqcap x \le y$
		\end{itemize}
		We call $\AA$ a \textbf{Boolean algebra} if moreover $1\le x\sqcup (x\sqsupset 0)$.
	\end{definition}
	\pause
	This generalises previous semantics as the meta-level truth values form such algebras!
\end{frame}

\newcommand{\lsem}{[\![}
\newcommand{\rsem}{]\!]}
\renewcommand{\LL}{\mathcal{L}}

\begin{frame}{Soundness}
	\pause
	Given any Heyting algebra $\AA$ and a valuation $V(P)\in \Omega$ for each atom define:
	\begin{align*}
		\lsem P\rsem &:= V(P)& \lsem A\land B\rsem &:= \lsem A\rsem \sqcap \lsem B\rsem\\
		\lsem \bot\rsem &:= 0& \lsem A\lor B\rsem &:= \lsem A\rsem \sqcup \lsem B\rsem\\
		\lsem \top\rsem &:= 1& \lsem A\to B\rsem &:= \lsem A\rsem \sqsupset \lsem B\rsem
	\end{align*}
	\vspace{-0.5cm}
	\pause
	\begin{theorem}[Soundness]
		If $\vdash A$ (or $\vdash_c A$), then $1\le \lsem A\rsem$ in any Heyting (or Boolean) algebra.
	\end{theorem}
	\vspace{-0.2cm}
	\pause
	\begin{proof}
		Show by induction on a derivation $\Gamma\vdash A$ that $\lsem \bigwedge\Gamma\rsem \le \lsem A\rsem$.
	\end{proof}
	\pause
	\begin{corollary}
		In NJ the formula $A\lor \neg A$ is not derivable.
	\end{corollary}
	\vspace{-0.2cm}
	\pause
	\begin{proof}
		Consider the linearly-ordered three-valued Heyting algebra $(0,\star,1)$.
	\end{proof}
\end{frame}

\begin{frame}{Completeness}
	\pause
	The completeness for algebraic semantics follows from a simple observation:
	\pause
	\begin{fact}[Lindenbaum Algebra]
		The formulas preorderd by $A\vdash B$ (or $A\vdash_c B$) form a Heyting (or Boolean) algebra.
	\end{fact}
	\vspace{-0.2cm}
	\pause
	\begin{proof}
		\begin{itemize}
			\pause
			\item
			Reflexivity by Ax rule, transitivity by IE rule, minimality of $\bot$ by Exp.
			\pause
			\item
			The equivalences for $\sqcap$, $\sqcup$ and $\sqsupset$ reduce to simple ND derivations.
			\qedhere
		\end{itemize}
	\end{proof}
	\pause
	\begin{corollary}[Completeness]
		If $1\le \lsem A\rsem$ in any Heyting (or Boolean) algebra, then $\vdash A$ (or $\vdash_c A$).
	\end{corollary}
	\vspace{-0.2cm}
	\pause
	\begin{proof}
		In the respective Lindenbaum algebra $1\le \lsem A\rsem$ immediately yields derivability of $A$.
	\end{proof}
\end{frame}

\begin{frame}{Complete Algebras}
	\pause
	We extend the algebraic semantics to first-order quantifiers:
	\pause
	\begin{definition}
		A Heyting algebra $\AA$ is called \textbf{complete} if it has arbitrary greatest lower bounds
		$$ \textstyle y\le \bigsqcap S  \text{ if and only if } y\le x \text{ for all } y\in S$$
		and then $\AA$ also has arbitrary least upper bounds $\bigsqcup S:=\bigsqcap\{x \mid \forall y\in S.\, y\le x\}$.
	\end{definition}
	\pause
	Given a complete Heyting algebra, we can interpret quantifiers in a natural way:
	$$\textstyle \lsem \forall x A\rsem :=\bigsqcap_t \lsem A[t/x]\rsem \hspace{5em}\lsem \exists x A\rsem :=\bigsqcup_t \lsem A[t/x]\rsem$$
	
	\pause
	\begin{itemize}
		\item
		Soundness: show by induction on $\Gamma\vdash A$ that $\lsem \bigwedge\Gamma[\sigma]\rsem \le \lsem A[\sigma]\rsem$
		\pause
		\item
		Completeness: needs more work as the Lindenbaum algebras are not complete
	\end{itemize}
\end{frame}

\begin{frame}{Mac-Neille Completion}
	\pause
	Luckily, there is a general recipe how algebras can be completed!
	\pause
	\begin{fact}
		Every Heyting algebra $\AA$ embeds into a complete Heyting algebra $\overline \AA$.
	\end{fact}
	\pause
	\begin{proof}
		\begin{enumerate}
			\pause
			\item
			Define $L(S):= \{ x\in \Omega\mid \forall y\in S.\,x\le y\}$ and $U(S):= \{ x\in \Omega\mid \forall y\in S.\,y\le x\}$.
			\pause
			\item
			Consider the set $\overline \Omega:=\{S\subseteq \Omega\mid L(U(S))\subseteq S\}$ ordered by inclusion.
			\pause
			\item
			Equip $\overline A$ with algebraic structure as follows:
			\begin{align*}
				\overline 0 &:=\{x\in \Omega\mid x\le 0\}&S\overline\sqcap S'&:=S\cap S' \\
				S\overline\sqsupset S'&:=\{x\in \Omega\mid \forall y\in S.\,x\sqcup y\in S'\}&S\overline\sqcup S'&:=L(U(S\cup S'))
			\end{align*}
			\item
			\pause
			Observe that $\overline \Omega$ is complete by using intersection $\bigcap$ (and union $\bigcup$).
			\pause
			\item
			Verify that $\overline x :=\{y\in \Omega\mid y\le x \}$ embeds $\AA$ into $\overline \AA$.
			\qedhere
		\end{enumerate}
	\end{proof}
\end{frame}

\begin{frame}{Completeness}
	\pause
	\begin{theorem}[Completeness]
		If $1\le \lsem A\rsem$ in any complete Heyting (or Boolean) algebra, then $\vdash A$ (or $\vdash_c A$).
	\end{theorem}
	\begin{proof}
		\begin{enumerate}
			\pause
			\item
			Assume that $\overline 1\subseteq \lsem A\rsem$ in the completion $\overline \LL$ of the Lindenbaum algebra for $\vdash A$.
			\pause
			\item
			Verify that evaluation in $\overline \LL$ coincides with the embedding of $\LL$: $\lsem A\rsem = \overline{ A}$.
			\pause
			\item
			But then in particular $\overline1 \subseteq \overline{A}$ and hence $1\le A$, meaning $\vdash A$.
			\pause
			\item
			To cover $\vdash_c A$, observe that the completion of Boolean algebras is Boolean.
			\qedhere
		\end{enumerate}
	\end{proof}
	
	\pause
	\vspace{0.3cm}
	Note that all algebraic completeness proofs as presented here are constructive!
\end{frame}

\begin{frame}{Heyting Arithmetic and Peano Arithmetic}
	\pause
	In the language of arithmetics $(0,S,+,\cdot)$ with equality consider the following axioms
	\begin{align*}
		x+0&=x & x+ (S y)&= S(x + y)\\
		x\cdot 0&= 0 & x \cdot (Sy) &= x+ x\cdot y\\
		0&\not= Sx & x &= y \text{ whenever } Sx = Sy
	\end{align*}
	as well as the following axiom scheme for every formula A:
	$$A(0) \to (\forall x A(x) \to A (Sx)) \to \forall x A(x)$$
	\pause
	\begin{definition}
		Heyting arithmetic (HA) consists of the above axioms and is based on the system NJ, Peano arithmetic (PA) consists of the same axioms but is based on NK.
	\end{definition}
\end{frame}

\begin{frame}{The Standard Model}
	\pause
	\begin{definition}
		We refer to the set $\mathbb N$ of natural numbers equipped with the arithmetical operations as the \textbf{standard model} of HA and PA.
	\end{definition}
	
	\pause
	\begin{fact}
		$\mathbb N$ satisfies all axioms of HA, so whenever $HA\vdash A$ also $\mathbb N\vDash A$.
	\end{fact}
	
	\pause
	\vspace{0.3cm}
	Due to the compactness theorem, there are also many non-standard models:
	\begin{enumerate}
		\pause
		\item
		Extend the language with a constant $c$ and consider axioms $\overline n\le c$ for all $n\in \mathbb N$.
		\pause
		\item
		Every finite set $\Gamma$ of these axioms is satisfied by $\mathbb N$ by choosing $c$ large enough.
		\pause
		\item
		Then by compactness there is a model $\MM$ satisfying all these axioms at once.
		\pause
		\item
		Observe that $\MM$ differs from $\mathbb N$ as $c$ in is larger than every standard number $\overline n$.
	\end{enumerate}
\end{frame}

\begin{frame}{Meta-Theoretic Properties}
	\pause
	Some standard (but potentially confusing) meta-theoretic results are:
	\begin{itemize}
		\pause
		\item
		Completeness: if $\MM\vDash A$ for all models $\MM$ of PA, then $PA\vdash_c A$
		\pause
		\item
		Incompleteness 1: if $\mathbb N \vDash A$, then not necessarily $PA\vdash_c A$
		\pause
		\item
		Incompleteness $1'$: there are sentences A with neither $PA\vdash_c A$ nor $PA\vdash_c \neg A$
		\pause
		\item
		Incompleteness 2: $PA\not\vdash_c Con(PA)$ expressing consistency of PA
	\end{itemize}
	
	\pause
	\vspace{0.5cm}
	Constructively, there are some even more interesting phenomena:
	\begin{itemize}
		\pause
		\item
		$\mathbb N$ is the only model of HA, a consequence of Tennenbaum's theorem
		\pause
		\item
		PA does not have a model at all, while it is still provably consistent
	\end{itemize}
\end{frame}

\begin{frame}{Friedman's A-Translation: Definition}
	\pause
	Observe that the double-negation translation $A^*$ can be verified in minimal logic:
	$$\vdash_c A\hspace{3em}\text{ if and only if }\hspace{3em}\vdash_m A^{*}$$
	
	\pause
	Therefore we can generalise the translation to an arbitrary formula $A_\bot$
	\begin{align*}
		P^{A_\bot} &:= \sim\sim \!\! P & (A\lor B)^{A_\bot} &:=\sim\sim \!\! (A^{A_\bot}\lor B^{A_\bot})\\
		(A\to B)^{A_\bot} &:= A^{A_\bot}\to B^{A_\bot} & (\forall x A)^{A_\bot} &:=\forall  x A^{A_\bot}\\
		(A\land B)^{A_\bot} &:=A^{A_\bot}\land B^{A_\bot} & (\exists x A)^{A_\bot} & := \sim\sim \!\! \exists x A^{A_\bot}
	\end{align*}
	where $\sim \!\! A$ is a shorthand for $A\to A_\bot$.
		
	\pause
	\vspace{0.3cm}
	\pause
	\begin{theorem}[Friedman]
		For arbitrary $A$ and $A_\bot$, we have that $\vdash_c A$ implies $\vdash_m A^{A_\bot}$.
	\end{theorem}
\end{frame}

\begin{frame}{Friedman's A-Translation: Proof Outline}
	\begin{enumerate}
		\pause
		\item
		Double-negation elimination holds for translated formulas: $\vdash_m \sim\sim\!\! A^{A_\bot}\to A^{A_\bot}$
		\begin{itemize}
			\pause
			\item
			By induction on $A$, exploiting that minimal logic identifies $\sim\sim\sim\!\! B$ and $\sim\!\! B$
		\end{itemize}
		\vspace{0.7cm}
		\pause
		\item
		Derivations can be replayed in the translation: $\Gamma\vdash_c A$ implies $\Gamma^{A_\bot}\vdash_m A^{A_\bot}$
		\begin{itemize}
			\pause
			\item
			By induction on $\Gamma\vdash_c A$, using (1) whenever negative reasoning is needed
		\end{itemize}
	\end{enumerate}
	
	\pause
	\vspace{0.9cm}
	Note however that it is \textbf{not} the case that classically $A$ and $A^{A_\bot}$ are always equivalent:\\
	\pause
	If we choose ${A_\bot}:=\top$ then we have $\vdash_c \bot^{A_\bot}$ but not $\vdash_c \bot$!
\end{frame}

\begin{frame}{Partial Conservativity Results}
	\pause
	The A-Translation can be used to show that some derivations of PA work also in HA!
	
	\pause
	\begin{theorem}
		Let $t(x)$ be given and assume $HA\vdash \neg \neg \exists x \, t(x)= 0$.
		Then $HA\vdash \exists x\, t(x)= 0$.
	\end{theorem}
	\pause
	\vspace{-0.3cm}
	\begin{proof}
		\begin{enumerate}
			\pause
			\item
			$HA\vdash \neg \neg \exists x \, t(x)= 0$ in particular $PA\vdash_c \exists x \, t(x)= 0$.
			\pause
			\item
			Choosing $A_\bot:= \exists x \, t(x)= 0$, we have $PA^{A_\bot}\vdash (\exists x \, t(x)= 0)^{A_\bot}$.
			\pause
			\item
			We can derive $HA\vdash PA^{A_\bot}$, so we have $HA\vdash (\exists x \, t(x)= 0)^{A_\bot}$.
			\pause
			\item
			From this we can derive $HA\vdash \exists x\, t(x)= 0$ as desired.
			\qedhere
		\end{enumerate}
	\end{proof}
	
	\pause
	\vspace{0.5cm}
	More generally for sentences $A$ of the form $\forall x \exists y\,B(x,y)$ with $B$ quantifier-free:
	\begin{itemize}
		\pause
		\item
		If A is provable in PA, then A is already provable in HA
		\pause
		\item
		If A is provable in ZF, then A is provable in IZF (intuitionistic Zermelo-Fraenkel)
	\end{itemize}
\end{frame}




\end{document}

%\begin{frame}{Bibliography}
%%\begin{frame}{Bibliography}
%\footnotesize
%\bibliographystyle{apalike}
%\bibliography{../biblio}
%\end{frame}

\appendix









\end{document}
