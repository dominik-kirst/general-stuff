 \documentclass[xcolor=dvipsnames,aspectratio=169,handout]{beamer}
\usecolortheme{orchid}
\setbeamertemplate{footline}[frame number]


\usepackage[utf8]{inputenc}
\usepackage[english]{babel}
\usepackage[OT1]{fontenc}
\usepackage{hyperref}
\usepackage[inference]{semantic}
\usepackage{graphicx}
\usepackage{amsfonts,amsmath}
\usepackage{cleveref}
\usepackage{scalerel}
\usepackage{multicol}
\usepackage{xspace}
\usepackage{amsmath}
\usepackage{amssymb}
\usepackage{stmaryrd}
\usepackage{proof}
\usepackage{natbib}
\usepackage{booktabs}

\usepackage{mathtools}
\usepackage{mathpartir}
\usepackage{colonequals}
\usepackage{xspace}
\usepackage{tabularx}
% Commands for correct abbreviations


\usepackage{graphicx}
\usepackage{bussproofs}
\usepackage{tikz}
\usetikzlibrary{shapes,arrows,chains}

\EnableBpAbbreviations

\usepackage{tikz}
\usepackage{tikz-cd}
\usetikzlibrary{calc}
\usetikzlibrary{arrows.meta}
\def\checkmark{\tikz\fill[scale=0.4](0,.35) -- (.25,0) -- (1,.7) -- (.25,.15) -- cycle;}

\usepackage{dsfont}

\title[Simply Typed Lambda Calculus 2]{Simply Typed Lambda Calculus 3}
\author[Dominik Kirst]{Dominik Kirst}

\date[December 2nd, 2025]{LMFI Proof Theory Course\\December 2nd, 2025}


% BOXES
\usepackage[many]{tcolorbox}
\newtcolorbox{emptybox}[1][]{
	beamer,
	width=(0.7\textwidth),
	% enlarge left by=-3pt,
	titlerule=0mm,
	colframe=white,
	coltitle=black,
	bottom=6pt, 
	top=-12pt,
	left=6pt,  
	right=6pt,
	notitle, 
	adjusted title={},
	outer arc=.5mm,
	arc=.5mm,
	no shadow,
	fuzzy shadow={1mm}{-1mm}{-1.2mm}{.7mm}{black!20},
	interior titled code={}
}

\newtcolorbox{widerbox}[1][]{
	beamer,
	width=(0.8\textwidth),
	% enlarge left by=-3pt,
	titlerule=0mm,
	colframe=white,
	coltitle=black,
	bottom=6pt, 
	top=-12pt,
	left=6pt,  
	right=6pt,
	notitle, 
	adjusted title={},
	outer arc=.5mm,
	arc=.5mm,
	no shadow,
	fuzzy shadow={1mm}{-1mm}{-1.2mm}{.7mm}{black!20},
	interior titled code={}
}

\newtcolorbox{widebox}[1][]{
	beamer,
	width=(\textwidth),
	% enlarge left by=-3pt,
	titlerule=0mm,
	colframe=white,
	coltitle=black,
	bottom=6pt, 
	top=-12pt,
	left=6pt,  
	right=6pt,
	notitle, 
	adjusted title={},
	outer arc=.5mm,
	arc=.5mm,
	no shadow,
	fuzzy shadow={1mm}{-1mm}{-1.2mm}{.7mm}{black!20},
	interior titled code={}
}


% TIMELINE
\usepackage{tikz}
\usepackage{graphicx}
\usepackage{stackengine}
\usetikzlibrary{trees, shapes, calc, positioning, shadows}

\definecolor{darkcerulean}{rgb}{0.063661, 0.257392, 0.477463}
\definecolor{scooter}{rgb}{0.161162, 0.775760, 0.885416}


% #1 – box name
% #2 – left color
% #3 – right color
% #4 – Title (may contain #1)
\newcommand{\newlargebox}[4]{
	\newtcolorbox{#1}[1][]{
		beamer,
		width=\textwidth+7pt,
		enlarge left by=-3pt,
		titlerule=3mm,
		colframe=white,
		coltitle={#2},
		bottom=2pt, 
		top=-4pt,
		left=6pt,  
		toptitle=2pt,
		bottomtitle=-2pt,
		fonttitle=\bfseries\large,
		adjusted title={#4},
		outer arc=.5mm,
		arc=.5mm,
		no shadow,
		fuzzy shadow={1mm}{-1mm}{-1.2mm}{.7mm}{black!20},
		interior titled code={
			\path [left color = {#2}, right color = {#3}]
			(title.south west) + (8pt, 0) rectangle ++(\textwidth-1pt, 0.02);
		}
	}
}


\newlargebox{defaultbox}{darkcerulean}{scooter}{#1} 

\usepackage{fancyvrb}
\usepackage{xcolor}


\begin{document}

%\beamerdefaultoverlayspecification{<+->}
\definecolor{red}{RGB}{204,0,0}
\definecolor{yellow}{RGB}{228,242,31}
\definecolor{green}{RGB}{0,204,0}

\newcommand\refs[1]{%
	\begin{textblock*}{8cm}(0.3cm,9.2cm)%
		\scriptsize {\color{gray}#1}
	\end{textblock*}
}

\renewcommand{\to}{\rightarrow}
\newcommand{\SN}{\textnormal{SN}}
\newcommand{\RED}{\textnormal{RED}}
\newcommand{\NE}{\textnormal{NE}}

\begin{frame}
	\maketitle
\end{frame}


\begin{frame}{Recap}
	\begin{itemize}
		\pause
		\item
		Proof terms:
		$$\infer{\vdash \lambda xy.x: A\Rightarrow B\Rightarrow A}{x:A,y:B\vdash x:A}\hspace{3em}\rightsquigarrow\hspace{3em}
		\infer{\vdash A\Rightarrow B\Rightarrow A}{A,B\vdash A}$$
		
		\pause
		\vspace{0.5cm}
		\item
		Strong normalisation:
		\begin{theorem}
			If $\Gamma\vdash t:\sigma$, then $t\in \RED_\sigma$.
		\end{theorem}
		\pause
		\vspace{0.5cm}
		\item
		Extension of STLC with products:
		$$
		\infer{\Gamma \vdash (t,u):\sigma\times \tau}{\Gamma\vdash t:\sigma&\Gamma\vdash u:\tau}
		\hspace{3em}
		\infer{\Gamma\vdash \pi_1 t : \sigma}{\Gamma\vdash t : \sigma \times \tau}
		\hspace{3em}
		\infer{\Gamma\vdash \pi_2 t : \tau}{\Gamma\vdash t : \sigma \times \tau}
		$$
	\end{itemize}
	
\end{frame}

\begin{frame}{Plan for Today}
	\begin{itemize}
		\item
		Extension of STLC with sums
		\begin{itemize}
		\item
		Syntax, operational semantics, normalisation, logical interpretation
		\end{itemize}
		\vspace{0.5cm}
		\item
		Gödel's system T
		\begin{itemize}
			\item
			Syntax, operational semantics, normalisation, logical interpretation
		\end{itemize}
		\vspace{0.5cm}
		\item
		Intersection type systems
		\begin{itemize}
			\item
			Connection to normalisation
		\end{itemize}
	\end{itemize}
\end{frame}

\newcommand{\match}[5]{\text{match }#1\text{ with }i_1{#2}\Rightarrow {#3} \text{ and } i_2{#4}\Rightarrow{#5}}
\newcommand{\newmatch}[5]{\delta#1(#2.#3)(#4.#5)}

\begin{frame}{Extending with Sums: Syntax}
	We extend the grammar for terms with primitive injections and matches
	$$t,u,v ::= \dots \mid i_1 t \mid i_2 t \mid \match t x u y v$$
	
	\pause
	and often write $\newmatch t x u y v$ instead of $\match t x u y v$.
	
	\pause
	\vspace{0.3cm}
	We then also extend the grammar for types with sum types:
	$$\sigma, \tau ::= \dots \mid \sigma + \tau$$
	
	\pause
	\vspace{0.3cm}
	We finally extend the typing judgement in the expected way:
	$$\infer{\Gamma\vdash i_1 t : \sigma+\tau}{\Gamma\vdash t : \sigma}
	\hspace{3em}
	\infer{\Gamma\vdash i_2 t : \sigma+\tau}{\Gamma\vdash t : \tau}
	\hspace{3em}
	\infer{\Gamma\vdash \newmatch t x u y v : \rho}{\Gamma\vdash t :\sigma+\tau &\Gamma,x:\sigma\vdash u:\rho&\Gamma,y:\sigma\vdash v:\rho}
	$$
	
\end{frame}

\begin{frame}{Extending with Products: Operational Semantics}
	We add two rules to the reduction system, telling how injections interact with matches:
	$$\newmatch {(i_1t)} x u y v\longrightarrow_\beta u\{t/x\}\hspace{3em}\newmatch {(i_2t)} x u y v\longrightarrow_\beta v\{t/x\}$$
	
	\pause
	\vspace{0.3cm}
	As with abstractions, sums can additionally be endowed with an extensionality law:
	$$\newmatch t x {i_1 x}y{i_2 y}\longrightarrow_\eta t$$
	
	\pause
	\vspace{0.3cm}
	In any case, the well-behaved structure is preserved:
	\begin{itemize}
		\item
		The extended reduction systems have the Church-Rosser property
		\item
		So in particular, normal forms are still unqiue
	\end{itemize}
\end{frame}

\begin{frame}{Extending with Products: Normalisation}
	\begin{enumerate}
		\pause
		\item
		Extend the previous proof, consider matches $\newmatch t x u y v$ neutral.
		\vspace{0.3cm}
		\pause
		\item
		Set $\RED_{\sigma+\tau}:= \{\Gamma \vdash t : \sigma+\tau \mid t\longrightarrow_\beta^* i_1 t',t'\in \RED_\sigma \lor t\longrightarrow_\beta^* i_2 t',t'\in \RED_\sigma \}$.
		\vspace{0.3cm}
		\pause
		\item
		For $t\in\RED_{\sigma+\tau}$ to $ t\in \SN$ establish the properties P1, P2 and P3.
		\vspace{0.3cm}
		\pause
		\item
		For $\Gamma \vdash t : \sigma+\tau$ to $ t\in\RED_{\sigma+\tau}$ establish the following lemma:
	\end{enumerate}
	\vspace{0.3cm}
	\pause
	\begin{lemma}
		If $t\in \RED_\sigma$ and $u\in \RED_\tau$, then $(t,u)\in \RED_{\sigma\times\tau}$.
	\end{lemma}
\end{frame}

\begin{frame}{Extending with Products: Logical Interpretation}
	Naturally, the typing rules for products correspond to the ND rules for conjunction:
	$$
	\infer{\Gamma \vdash (t,u):\sigma\times \tau}{\Gamma\vdash t:\sigma&\Gamma\vdash u:\tau}
	\hspace{3em}
	\infer{\Gamma\vdash \pi_1 t : \sigma}{\Gamma\vdash t : \sigma \times \tau}
	\hspace{3em}
	\infer{\Gamma\vdash \pi_2 t : \tau}{\Gamma\vdash t : \sigma \times \tau}
	$$
	
	\pause
	$$
	\infer{\Gamma \vdash A\land B}{\Gamma\vdash A&\Gamma\vdash B}
	\hspace{3em}
	\infer{\Gamma\vdash A}{\Gamma\vdash A\land B}
	\hspace{3em}
	\infer{\Gamma\vdash B}{\Gamma\vdash A\land B}
	$$
	
	\pause
	\vspace{0.5cm}
	So the Curry-Howard isomorphism extends to conjunction!
	\begin{itemize}
		\item
		$\lambda x.\pi_1 x$ is a proof term for $A\land B \Rightarrow A$
		\item 
		$\lambda x.\pi_2 x$ is a proof term for $A\land B \Rightarrow B$
		\item
		$\lambda xy.(x,y)$ is a proof term for $A\Rightarrow B\Rightarrow A\land B$
	\end{itemize}
\end{frame}




\end{document}



%\begin{frame}{Bibliography}
%%\begin{frame}{Bibliography}
%\footnotesize
%\bibliographystyle{apalike}
%\bibliography{../biblio}
%\end{frame}

\appendix









\end{document}
