 \documentclass[xcolor=dvipsnames,aspectratio=169,handout]{beamer}
\usecolortheme{orchid}
\setbeamertemplate{footline}[frame number]


\usepackage[utf8]{inputenc}
\usepackage[english]{babel}
\usepackage[OT1]{fontenc}
\usepackage{hyperref}
\usepackage[inference]{semantic}
\usepackage{graphicx}
\usepackage{amsfonts,amsmath}
\usepackage{cleveref}
\usepackage{scalerel}
\usepackage{multicol}
\usepackage{xspace}
\usepackage{amsmath}
\usepackage{amssymb}
\usepackage{stmaryrd}
\usepackage{proof}
\usepackage{natbib}
\usepackage{booktabs}

\usepackage{mathtools}
\usepackage{mathpartir}
\usepackage{colonequals}
\usepackage{xspace}
\usepackage{tabularx}
% Commands for correct abbreviations


\usepackage{graphicx}
\usepackage{bussproofs}
\usepackage{tikz}
\usetikzlibrary{shapes,arrows,chains}

\EnableBpAbbreviations

\usepackage{tikz}
\usepackage{tikz-cd}
\usetikzlibrary{calc}
\usetikzlibrary{arrows.meta}
\def\checkmark{\tikz\fill[scale=0.4](0,.35) -- (.25,0) -- (1,.7) -- (.25,.15) -- cycle;}

\usepackage{dsfont}

\title[Simply Typed Lambda Calculus 2]{Simply Typed Lambda Calculus 3}
\author[Dominik Kirst]{Dominik Kirst}

\date[December 2nd, 2025]{LMFI Proof Theory Course\\December 2nd, 2025}


% BOXES
\usepackage[many]{tcolorbox}
\newtcolorbox{emptybox}[1][]{
	beamer,
	width=(0.7\textwidth),
	% enlarge left by=-3pt,
	titlerule=0mm,
	colframe=white,
	coltitle=black,
	bottom=6pt, 
	top=-12pt,
	left=6pt,  
	right=6pt,
	notitle, 
	adjusted title={},
	outer arc=.5mm,
	arc=.5mm,
	no shadow,
	fuzzy shadow={1mm}{-1mm}{-1.2mm}{.7mm}{black!20},
	interior titled code={}
}

\newtcolorbox{widerbox}[1][]{
	beamer,
	width=(0.8\textwidth),
	% enlarge left by=-3pt,
	titlerule=0mm,
	colframe=white,
	coltitle=black,
	bottom=6pt, 
	top=-12pt,
	left=6pt,  
	right=6pt,
	notitle, 
	adjusted title={},
	outer arc=.5mm,
	arc=.5mm,
	no shadow,
	fuzzy shadow={1mm}{-1mm}{-1.2mm}{.7mm}{black!20},
	interior titled code={}
}

\newtcolorbox{widebox}[1][]{
	beamer,
	width=(\textwidth),
	% enlarge left by=-3pt,
	titlerule=0mm,
	colframe=white,
	coltitle=black,
	bottom=6pt, 
	top=-12pt,
	left=6pt,  
	right=6pt,
	notitle, 
	adjusted title={},
	outer arc=.5mm,
	arc=.5mm,
	no shadow,
	fuzzy shadow={1mm}{-1mm}{-1.2mm}{.7mm}{black!20},
	interior titled code={}
}


% TIMELINE
\usepackage{tikz}
\usepackage{graphicx}
\usepackage{stackengine}
\usetikzlibrary{trees, shapes, calc, positioning, shadows}

\definecolor{darkcerulean}{rgb}{0.063661, 0.257392, 0.477463}
\definecolor{scooter}{rgb}{0.161162, 0.775760, 0.885416}


% #1 – box name
% #2 – left color
% #3 – right color
% #4 – Title (may contain #1)
\newcommand{\newlargebox}[4]{
	\newtcolorbox{#1}[1][]{
		beamer,
		width=\textwidth+7pt,
		enlarge left by=-3pt,
		titlerule=3mm,
		colframe=white,
		coltitle={#2},
		bottom=2pt, 
		top=-4pt,
		left=6pt,  
		toptitle=2pt,
		bottomtitle=-2pt,
		fonttitle=\bfseries\large,
		adjusted title={#4},
		outer arc=.5mm,
		arc=.5mm,
		no shadow,
		fuzzy shadow={1mm}{-1mm}{-1.2mm}{.7mm}{black!20},
		interior titled code={
			\path [left color = {#2}, right color = {#3}]
			(title.south west) + (8pt, 0) rectangle ++(\textwidth-1pt, 0.02);
		}
	}
}


\newlargebox{defaultbox}{darkcerulean}{scooter}{#1} 

\usepackage{fancyvrb}
\usepackage{xcolor}


\begin{document}

%\beamerdefaultoverlayspecification{<+->}
\definecolor{red}{RGB}{204,0,0}
\definecolor{yellow}{RGB}{228,242,31}
\definecolor{green}{RGB}{0,204,0}

\newcommand\refs[1]{%
	\begin{textblock*}{8cm}(0.3cm,9.2cm)%
		\scriptsize {\color{gray}#1}
	\end{textblock*}
}

\renewcommand{\to}{\rightarrow}
\newcommand{\SN}{\textnormal{SN}}
\newcommand{\RED}{\textnormal{RED}}
\newcommand{\NE}{\textnormal{NE}}

\begin{frame}
	\maketitle
\end{frame}


\begin{frame}{Recap}
	\begin{itemize}
		\pause
		\item
		Proof terms:
		$$\infer{\vdash \lambda xy.x: A\Rightarrow B\Rightarrow A}{x:A,y:B\vdash x:A}\hspace{3em}\rightsquigarrow\hspace{3em}
		\infer{\vdash A\Rightarrow B\Rightarrow A}{A,B\vdash A}$$
		
		\pause
		\vspace{0.5cm}
		\item
		Strong normalisation:
		\begin{theorem}
			If $\Gamma\vdash t:\sigma$, then $t\in \SN$.
		\end{theorem}
		\pause
		\vspace{0.5cm}
		\item
		Extension of STLC with products:
		$$
		\infer{\Gamma \vdash (t,u):\sigma\times \tau}{\Gamma\vdash t:\sigma&\Gamma\vdash u:\tau}
		\hspace{3em}
		\infer{\Gamma\vdash \pi_1 t : \sigma}{\Gamma\vdash t : \sigma \times \tau}
		\hspace{3em}
		\infer{\Gamma\vdash \pi_2 t : \tau}{\Gamma\vdash t : \sigma \times \tau}
		$$
	\end{itemize}
	
\end{frame}

\begin{frame}{Plan for Today}
	\begin{itemize}
		\item
		Extension of STLC with sums
		\begin{itemize}
		\item
		Syntax, operational semantics, normalisation, logical interpretation
		\end{itemize}
		\vspace{0.5cm}
		\item
		Gödel's system T
		\begin{itemize}
			\item
			Syntax, operational semantics, normalisation, realisability interpretation
		\end{itemize}
		\vspace{0.5cm}
		\item
		Intersection type systems
		\begin{itemize}
			\item
			Connection to normalisation
		\end{itemize}
	\end{itemize}
\end{frame}

\newcommand{\match}[5]{\text{match }#1\text{ with }\iota_1{#2}\Rightarrow {#3} \text{ and } \iota_2{#4}\Rightarrow{#5}}
\newcommand{\newmatch}[5]{\delta#1(#2.#3)(#4.#5)}

\begin{frame}{Extending with Sums: Syntax}
	We extend the grammar for terms with primitive injections and matches
	$$t,u,v ::= \dots \mid \iota_1 t \mid \iota_2 t \mid \match t x u y v$$
	
	\pause
	and often write $\newmatch t x u y v$ instead of $\match t x u y v$.
	
	\pause
	\vspace{0.3cm}
	We then also extend the grammar for types with sum types:
	$$\sigma, \tau ::= \dots \mid \sigma + \tau$$
	
	\pause
	\vspace{0.3cm}
	We finally extend the typing judgement in the expected way:
	$$\infer{\Gamma\vdash \iota_1 t : \sigma+\tau}{\Gamma\vdash t : \sigma}
	\hspace{3em}
	\infer{\Gamma\vdash \iota_2 t : \sigma+\tau}{\Gamma\vdash t : \tau}
	\hspace{3em}
	\infer{\Gamma\vdash \newmatch t x u y v : \upsilon}{\Gamma\vdash t :\sigma+\tau &\Gamma,x:\sigma\vdash u:\upsilon&\Gamma,y:\sigma\vdash v:\upsilon}
	$$
	
\end{frame}

\begin{frame}{Extending with Sums: Operational Semantics}
	We add two rules to the reduction system, telling how injections interact with matches:
	$$\newmatch {(\iota_1t)} x u y v\longrightarrow_\beta u\{t/x\}\hspace{3em}\newmatch {(\iota_2t)} x u y v\longrightarrow_\beta v\{t/y\}$$
	
	\pause
	\vspace{0.3cm}
	As with abstractions, sums can additionally be endowed with an extensionality law:
	$$\newmatch t x {\iota_1 x}y{\iota_2 y}\longrightarrow_\eta t$$
	
	\pause
	\vspace{0.3cm}
	In any case, the well-behaved structure is preserved:
	\begin{itemize}
		\item
		The extended reduction systems have the Church-Rosser property
		\item
		So in particular, normal forms are still unqiue
	\end{itemize}
\end{frame}

\begin{frame}{Extending with Sums: Normalisation}
	\begin{enumerate}
		\pause
		\item
		Extend the previous proof, consider matches $\newmatch t x u y v$ neutral.
		\vspace{0.3cm}
		\pause
		\item
		Define $\RED_{\sigma+\tau}:= \{\Gamma \vdash t : \sigma+\tau \mid t\in \SN \land \forall t'.\, ((t\longrightarrow_\beta^* \iota_1 t')\to t'\in \RED_\sigma)$\\ \hspace{19.51em} $\land ((t\longrightarrow_\beta^* \iota_2 t')\to t'\in \RED_\tau) \}$.
		\vspace{0.3cm}
		\pause
		\item
		For $t\in\RED_{\sigma+\tau}$ establish the properties P1, P2 and P3.
		\vspace{0.3cm}
		\pause
		\item
		For $\Gamma \vdash t : \sigma+\tau$ to $ t\in\RED_{\sigma+\tau}$ establish the following lemma:
	\end{enumerate}
	\vspace{0.3cm}
	\pause
	\begin{lemma}
		If $t\in \RED_{\sigma+\tau}$ and, additionally, for all $u'\in \RED_\sigma$ we have $u\{u'/x \}\in \RED_\upsilon$ and for all $v'\in \RED_\tau$ we have $v\{v'/y \}\in \RED_\upsilon$, then $\newmatch t x u y v\in \RED_{\upsilon}$.
	\end{lemma}
\end{frame}

\begin{frame}{Extending with Sums: Logical Interpretation}
	Naturally, the typing rules for products correspond to the ND rules for disjunction:
	$$\infer{\Gamma\vdash \iota_1 t : \sigma+\tau}{\Gamma\vdash t : \sigma}
	\hspace{3em}
	\infer{\Gamma\vdash \iota_2 t : \sigma+\tau}{\Gamma\vdash t : \tau}
	\hspace{3em}
	\infer{\Gamma\vdash \newmatch t x u y v : \upsilon}{\Gamma\vdash t :\sigma+\tau &\Gamma,x:\sigma\vdash u:\upsilon&\Gamma,y:\sigma\vdash v:\upsilon}
	$$
	
	\pause
	$$
	\infer{\Gamma\vdash A\lor B}{\Gamma\vdash A}
	\hspace{3em}
	\infer{\Gamma\vdash A\lor B}{\Gamma\vdash B}
	\hspace{3em}
	\infer{\Gamma\vdash C}{\Gamma\vdash A\lor B&\Gamma,A\vdash C&\Gamma,B\vdash C}
	$$
	
	\pause
	\vspace{0.5cm}
	So the Curry-Howard isomorphism extends to disjunction!
	\begin{itemize}
		\item
		$\lambda x.\iota_1 x$ is a proof term for $A \Rightarrow A\lor B$
		\item 
		$\lambda x.\iota_2 x$ is a proof term for $B \Rightarrow A\lor B$
		\item
		$\lambda zuv.\newmatch z x {ux} y{vy}$ is a proof term for $A\lor B \Rightarrow (A\Rightarrow C) \Rightarrow (B\Rightarrow C)\Rightarrow C$
	\end{itemize}
\end{frame}

\renewcommand{\O}{\mathsf{O}}
\renewcommand{\S}{\mathsf{S}}
\newcommand{\R}{\mathsf{R}}

\begin{frame}{System T: Syntax}
	We extend the grammar for terms with primitive injections and matches
	$$t,u,v ::= \dots \mid \O\mid \S\,t\mid \R\,u\,v\,t .$$
	
	\pause
	\vspace{0.3cm}
	We then also extend the grammar for types with a type of numbers:
	$$\sigma, \tau ::= \dots \mid \omega$$
	
	\pause
	\vspace{0.3cm}
	We finally extend the typing judgement in the expected way:
	$$\infer{\Gamma\vdash O : \omega}{}
	\hspace{3em}
	\infer{\Gamma\vdash \S\, t : \omega}{\Gamma\vdash t : \omega}
	\hspace{3em}
	\infer{\Gamma\vdash \R\,u\,v\,t:\sigma}{\Gamma\vdash u :\sigma &\Gamma\vdash v:\sigma \to \omega\to \sigma&\Gamma\vdash t:\omega}
	$$
	
\end{frame}

\begin{frame}{System T: Operational Semantics}
	We add two rules to the reduction system, telling how the recursor operates:
	$$\R\,u\,v\,\O\longrightarrow_\beta u\hspace{3em}\R\,u\,v\,(\S\,t)\longrightarrow_\beta v(\R\,u\,v\,t)t$$
	
	\pause
	\vspace{0.3cm}
	Then we can define functions recursively, for instance addition:
	$$t+t'~:=~\R\,t\,(\lambda xy.\,\S\,x)\,t'\hspace{5em}\overline 6+\overline 2\longrightarrow_\beta^* \overline 8$$
	
	\pause
	\vspace{0.3cm}
	In any case, the well-behaved structure is preserved:
	\begin{itemize}
		\item
		The extended reduction systems have the Church-Rosser property
		\item
		So in particular, normal forms are still unqiue
	\end{itemize}
\end{frame}

\begin{frame}{System T: Normalisation}
	\begin{enumerate}
		\pause
		\item
		Extend the previous proof, consider recursion $\R\,u\,v\,t$ neutral.
		\vspace{0.3cm}
		\pause
		\item
		Define $\RED_\omega:= \{\Gamma \vdash t : \omega \mid t\in \SN  \}$.
		\vspace{0.3cm}
		\pause
		\item
		For $t\in\RED_\omega$ establish the properties P1, P2 and P3.
		\vspace{0.3cm}
		\pause
		\item
		For $\Gamma \vdash t : \omega$ to $ t\in\RED_\omega$ establish the following lemma:
	\end{enumerate}
	\vspace{0.3cm}
	\pause
	\begin{lemma}
		If $t\in \RED_\omega$ then $\S\, t\in \RED_\omega$.
	\end{lemma}
\end{frame}

\renewcommand{\phi}{\varphi}

\begin{frame}{System T: Realisability Interpretation}
	\pause
	There is a systematic way how formulas of HA can be interpreted as types in System T:
	\begin{align*}
		T(n=m)&:=\omega& T(A_1\Rightarrow A_2 )&:=T(A_1)\to T(A_2 )\\
		T(A_1\land A_2)&:=T(A_1)\times T(A_2 )&T(\forall x A)&:= \omega \to T(A)\\
		T(A_1\lor A_2 )&:=T(A_1)+T(A_2 )&T(\exists x A)&:= \omega \times T(A)
	\end{align*}
	
	\pause
	Moreover, we can interpreters as their sets of realising terms of System T:
	\begin{align*}
		t\Vdash n=m &~:=~n \text{ equals } m& t\Vdash A_1 \Rightarrow A_2 &~:=~ tu\Vdash A_2  \text{ for all } u\Vdash A_1\\
		t\Vdash A_1 \land A_2  &~:=~\pi_1t \Vdash A_1 \text{ and } \pi_2t \Vdash A_2 & t\Vdash \forall x A &~:=~ t\overline n\Vdash A[\overline n/x] \text{ for all }n\in \mathbb{N} \\
		\iota_i t\Vdash A_1\lor A_2 &~:=~t\Vdash A_i& t\Vdash \exists x A&:= \pi_2 t\Vdash A[\pi_1 t/x]
	\end{align*}
	
	\pause
	\begin{theorem}
		If $HA\vdash A$ then there is $t:T(A)$ such that $t\Vdash A$.
	\end{theorem}
\end{frame}

\begin{frame}{Intersection Types: Overview}
	\pause
	There are many different systems, here is one possible version:
	$$\sigma,\tau::= A\mid \bigwedge_{i=1}^k\tau_i \to \sigma$$
	
	\vspace{0.3cm}
	\pause
	The typing relation is adjusted accordingly with the following rule:
	$$\infer{\Gamma \vdash tu\vdash \sigma}{\Gamma \vdash t  \bigwedge_{i=1}^k\tau_i \to \sigma&\forall i.\, (\Gamma \vdash u:\tau_i)}$$
	
	\vspace{0.3cm}
	\pause
	Note that with this choice for instance $xx$ and $KI\Omega$ are typeable!
\end{frame}

\begin{frame}{Intersection Types: Normalisation}
	\pause
	\begin{theorem}
		A term $t$ is typeable if and only if it has a normal form.
	\end{theorem}
	\pause
	\begin{proof}
		That typeable terms have normal forms is complicated, for the converse:
		\begin{enumerate}
			\item
			First show that normal forms are typeable.
			\item
			Establish subject expansion, showing that typeability propagates backwards.
			\item
			Conclude that all terms with normal forms are typeable.
			\qedhere
		\end{enumerate}
	\end{proof}
	
	\vspace{0.3cm}
	\pause
	Consequence: typeability is undecidable in this system!
\end{frame}



\end{document}



%\begin{frame}{Bibliography}
%%\begin{frame}{Bibliography}
%\footnotesize
%\bibliographystyle{apalike}
%\bibliography{../biblio}
%\end{frame}

\appendix









\end{document}
