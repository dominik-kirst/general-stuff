 \documentclass[xcolor=dvipsnames,aspectratio=169,handout]{beamer}
\usecolortheme{orchid}
\setbeamertemplate{footline}[frame number]


\usepackage[utf8]{inputenc}
\usepackage[english]{babel}
\usepackage[OT1]{fontenc}
\usepackage{hyperref}
\usepackage[inference]{semantic}
\usepackage{graphicx}
\usepackage{amsfonts,amsmath}
\usepackage{cleveref}
\usepackage{scalerel}
\usepackage{multicol}
\usepackage{xspace}
\usepackage{amsmath}
\usepackage{amssymb}
\usepackage{stmaryrd}
\usepackage{proof}
\usepackage{natbib}
\usepackage{booktabs}

\usepackage{mathtools}
\usepackage{mathpartir}
\usepackage{colonequals}
\usepackage{xspace}
\usepackage{tabularx}
% Commands for correct abbreviations


\usepackage{graphicx}
\usepackage{bussproofs}
\usepackage{tikz}
\usetikzlibrary{shapes,arrows,chains}

\EnableBpAbbreviations

\usepackage{tikz}
\usepackage{tikz-cd}
\usetikzlibrary{calc}
\usetikzlibrary{arrows.meta}
\def\checkmark{\tikz\fill[scale=0.4](0,.35) -- (.25,0) -- (1,.7) -- (.25,.15) -- cycle;}

\usepackage{dsfont}

\title[Simply Typed Lambda Calculus 2]{Simply Typed Lambda Calculus 2}
\author[Dominik Kirst]{Dominik Kirst}

\date[November 25th, 2025]{LMFI Proof Theory Course\\November 25th, 2025}


% BOXES
\usepackage[many]{tcolorbox}
\newtcolorbox{emptybox}[1][]{
	beamer,
	width=(0.7\textwidth),
	% enlarge left by=-3pt,
	titlerule=0mm,
	colframe=white,
	coltitle=black,
	bottom=6pt, 
	top=-12pt,
	left=6pt,  
	right=6pt,
	notitle, 
	adjusted title={},
	outer arc=.5mm,
	arc=.5mm,
	no shadow,
	fuzzy shadow={1mm}{-1mm}{-1.2mm}{.7mm}{black!20},
	interior titled code={}
}

\newtcolorbox{widerbox}[1][]{
	beamer,
	width=(0.8\textwidth),
	% enlarge left by=-3pt,
	titlerule=0mm,
	colframe=white,
	coltitle=black,
	bottom=6pt, 
	top=-12pt,
	left=6pt,  
	right=6pt,
	notitle, 
	adjusted title={},
	outer arc=.5mm,
	arc=.5mm,
	no shadow,
	fuzzy shadow={1mm}{-1mm}{-1.2mm}{.7mm}{black!20},
	interior titled code={}
}

\newtcolorbox{widebox}[1][]{
	beamer,
	width=(\textwidth),
	% enlarge left by=-3pt,
	titlerule=0mm,
	colframe=white,
	coltitle=black,
	bottom=6pt, 
	top=-12pt,
	left=6pt,  
	right=6pt,
	notitle, 
	adjusted title={},
	outer arc=.5mm,
	arc=.5mm,
	no shadow,
	fuzzy shadow={1mm}{-1mm}{-1.2mm}{.7mm}{black!20},
	interior titled code={}
}


% TIMELINE
\usepackage{tikz}
\usepackage{graphicx}
\usepackage{stackengine}
\usetikzlibrary{trees, shapes, calc, positioning, shadows}

\definecolor{darkcerulean}{rgb}{0.063661, 0.257392, 0.477463}
\definecolor{scooter}{rgb}{0.161162, 0.775760, 0.885416}


% #1 – box name
% #2 – left color
% #3 – right color
% #4 – Title (may contain #1)
\newcommand{\newlargebox}[4]{
	\newtcolorbox{#1}[1][]{
		beamer,
		width=\textwidth+7pt,
		enlarge left by=-3pt,
		titlerule=3mm,
		colframe=white,
		coltitle={#2},
		bottom=2pt, 
		top=-4pt,
		left=6pt,  
		toptitle=2pt,
		bottomtitle=-2pt,
		fonttitle=\bfseries\large,
		adjusted title={#4},
		outer arc=.5mm,
		arc=.5mm,
		no shadow,
		fuzzy shadow={1mm}{-1mm}{-1.2mm}{.7mm}{black!20},
		interior titled code={
			\path [left color = {#2}, right color = {#3}]
			(title.south west) + (8pt, 0) rectangle ++(\textwidth-1pt, 0.02);
		}
	}
}


\newlargebox{defaultbox}{darkcerulean}{scooter}{#1} 

\usepackage{fancyvrb}
\usepackage{xcolor}


\begin{document}

%\beamerdefaultoverlayspecification{<+->}
\definecolor{red}{RGB}{204,0,0}
\definecolor{yellow}{RGB}{228,242,31}
\definecolor{green}{RGB}{0,204,0}

\newcommand\refs[1]{%
	\begin{textblock*}{8cm}(0.3cm,9.2cm)%
		\scriptsize {\color{gray}#1}
	\end{textblock*}
}

\renewcommand{\to}{\rightarrow}

\begin{frame}
	\maketitle
\end{frame}


\begin{frame}{Recap}
	\begin{itemize}
		\pause
		\item
		Simply typed lambda calculus:
		$$
		\infer{\Gamma \vdash x : \sigma}{(x:\sigma)\in \Gamma}\hspace{0.8cm}
		\infer{\Gamma \vdash \lambda x^\sigma.t : \sigma\rightarrow \tau}{\Gamma, x:\sigma\vdash t:\tau}\hspace{0.8cm}
		\infer{\Gamma\vdash tu:\tau}{\Gamma\vdash t :\sigma\rightarrow \tau & \Gamma \vdash u :\sigma}
		$$
		\pause
		\item
		Structural properties:
		\begin{fact}[Subject Reduction]
			If $\Gamma\vdash t : \sigma$ and $t\longrightarrow_\beta u$, then $\Gamma\vdash u: \sigma$.
		\end{fact}
		\pause
		\item
		Normalisation:
		\begin{theorem}[Weak Normalisation]
			If $\Gamma\vdash t : \sigma$, then $t$ has a normal form.
		\end{theorem}
		\pause
		\item
		Curry-Howard isomorphism: identify propositions with types, proofs with programs
	\end{itemize}
	
\end{frame}

\begin{frame}{Plan for Today}
	\begin{itemize}
		\item
		More on the Curry-Howard isomorphism
		\begin{itemize}
		\item
		Explicit proof terms
		\end{itemize}
		\vspace{0.5cm}
		\item
		Strong normalisation
		\begin{itemize}
			\item
			Full proof in detail
		\end{itemize}
		\vspace{0.5cm}
		\item
		Extension with products
		\begin{itemize}
			\item
			Syntax, operational semantics, normalisation, logical interpretation
		\end{itemize}
	\end{itemize}
\end{frame}

\begin{frame}{Proof Terms}
	In the last lecture, we observed the Curry-Howard isomorphism by deleting terms:
	
	$$\infer{\vdash \lambda x^A.x :A\to A}{(x:A)\vdash x:A}
	\hspace{3em}\rightsquigarrow\hspace{3em}
	\infer{\vdash A\to A}{A\vdash A}
	$$
	
	\vspace{0.5cm}
	However, well-typed terms can be seen as proofs!
	\begin{itemize}
		\item
		$\lambda x.x$ is a proof term for $A\Rightarrow A$
		\item
		$\lambda xy.x$ is a proof term for $A\Rightarrow B\Rightarrow A$
		\item
		$ \lambda x y z. x z(yz)$ is a proof term for $(A\Rightarrow B\Rightarrow C)\Rightarrow (A\Rightarrow B) \Rightarrow A \Rightarrow C$
	\end{itemize}
\end{frame}

\begin{frame}{Strong Normalisation: Naive Ideas}
	content
\end{frame}

\begin{frame}{Strong Normalisation: Reducible Terms}
	content
\end{frame}

\begin{frame}{Strong Normalisation: Reducibility at Atomic Types}
	content
\end{frame}

\begin{frame}{Strong Normalisation: Reducibility at Function Types}
	content
\end{frame}

\begin{frame}{Strong Normalisation: Reduction Lemma}
	content
\end{frame}

\begin{frame}{Strong Normalisation: Conclusion}
	content
\end{frame}

\begin{frame}{Extending with Products: Syntax}
	We extend the grammar for terms with primitive pairs and projections:
	$$t,u ::= x\mid tu\mid \lambda x. t\mid (t,u)\mid \pi_1 t \mid \pi_2 t$$
	
	\vspace{0.3cm}
	We then also extend the grammar for types with product types:
	$$\sigma, \tau ::= A \mid \sigma\to\tau \mid \sigma \times \tau$$
	
	\vspace{0.3cm}
	We finally extend the typing judgement in the expected way:
	$$\infer{\Gamma \vdash (t,u):\sigma\times \tau}{\Gamma\vdash t:\sigma&\Gamma\vdash u:\tau}
	\hspace{3em}
	\infer{\Gamma\vdash \pi_1 t : \sigma}{\Gamma\vdash t : \sigma \times \tau}
	\hspace{3em}
	\infer{\Gamma\vdash \pi_2 t : \tau}{\Gamma\vdash t : \sigma \times \tau}
	$$
	
\end{frame}

\begin{frame}{Extending with Products: Operational Semantics}
	We add two rules to the reduction system, telling how projections interact with pairs:
	$$\pi_1(t,u)\longrightarrow_\beta t\hspace{3em}\pi_2(t,u)\longrightarrow_\beta u$$
	
	\vspace{0.3cm}
	As with abstractions, pairs can additionally be endowed with an extensionality law:
	$$(\pi_1 t,\pi_2 t)\longrightarrow_\eta t$$
	
	\vspace{0.3cm}
	In any case, the well-behaved structure is preserved:
	\begin{itemize}
		\item
		The extended reduction systems have the Church-Rosser property
		\item
		So in particular, normal forms are still unqiue
	\end{itemize}
\end{frame}

\begin{frame}{Extending with Products: Normalisation}
	content
\end{frame}

\begin{frame}{Extending with Products: Logical Interpretation}
	Naturally, the typing rules for products correspond to the ND rules for conjunction:
	$$
	\infer{\Gamma \vdash (t,u):\sigma\times \tau}{\Gamma\vdash t:\sigma&\Gamma\vdash u:\tau}
	\hspace{3em}
	\infer{\Gamma\vdash \pi_1 t : \sigma}{\Gamma\vdash t : \sigma \times \tau}
	\hspace{3em}
	\infer{\Gamma\vdash \pi_2 t : \tau}{\Gamma\vdash t : \sigma \times \tau}
	$$
	
	\pause
	$$
	\infer{\Gamma \vdash A\land B}{\Gamma\vdash A&\Gamma\vdash B}
	\hspace{3em}
	\infer{\Gamma\vdash A}{\Gamma\vdash A\land B}
	\hspace{3em}
	\infer{\Gamma\vdash B}{\Gamma\vdash A\land B}
	$$
	
	\vspace{0.5cm}
	So the Curry-Howard isomorphism extends to conjunction!
	\begin{itemize}
		\item
		$\lambda x.\pi_1 x$ is a proof term for $A\land B \Rightarrow A$
		\item 
		$\lambda x.\pi_2 x$ is a proof term for $A\land B \Rightarrow B$
		\item
		$\lambda xy.(x,y)$ is a proof term for $A\Rightarrow B\Rightarrow A\land B$
	\end{itemize}
\end{frame}



\end{document}



%\begin{frame}{Bibliography}
%%\begin{frame}{Bibliography}
%\footnotesize
%\bibliographystyle{apalike}
%\bibliography{../biblio}
%\end{frame}

\appendix









\end{document}
