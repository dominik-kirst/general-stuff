\documentclass{article}
\usepackage[left=3cm,right=3cm,top=3cm,bottom=3cm]{geometry}

\usepackage[utf8]{inputenc}
\usepackage[english]{babel}
\usepackage[OT1]{fontenc}
\usepackage{hyperref}
\usepackage[inference]{semantic}
\usepackage{graphicx}
\usepackage{amsfonts,amsmath}
\usepackage{cleveref}
\usepackage{scalerel}
\usepackage{multicol}
\usepackage{xspace}
\usepackage{amsmath}
\usepackage{amssymb}
\usepackage{stmaryrd}
\usepackage{proof}
\usepackage{natbib}
\usepackage{booktabs}

\usepackage{mathtools}
\usepackage{mathpartir}
\usepackage{colonequals}
\usepackage{xspace}
% Commands for correct abbreviations


\usepackage{graphicx}
\usepackage{bussproofs}
\usepackage{tikz}
\usetikzlibrary{shapes,arrows,chains}

\EnableBpAbbreviations

\usepackage{tikz}
\usepackage{tikz-cd}
\usetikzlibrary{calc}
\usetikzlibrary{arrows.meta}
\def\checkmark{\tikz\fill[scale=0.4](0,.35) -- (.25,0) -- (1,.7) -- (.25,.15) -- cycle;}

\usepackage{dsfont}

\renewcommand{\to}{\Rightarrow}
\newcommand{\TT}{\mathcal{T}}
\newcommand{\MM}{\mathcal{M}}
\newcommand{\KK}{\mathcal{K}}
\newcommand{\UU}{\mathcal U}
\renewcommand{\AA}{\mathcal A}
\newcommand{\lsem}{[\![}
\newcommand{\rsem}{]\!]}
\renewcommand{\LL}{\mathcal{L}}

\newtheorem{exercise}{Question}
\newtheorem{definition}{Definition}
\newtheorem{theorem}{Theorem}

\begin{document}
	
\newcommand{\SN}{\textnormal{SN}}
\newcommand{\RED}{\textnormal{RED}}
\newcommand{\NE}{\textnormal{NE}}

\title{LMFI – Cours fondamental de logique\\Théorie de la démonstration}
\author{Part on Natural Deduction}

\date{Dominik Kirst\\End-term exam, December 17th, 2025}

\maketitle

\noindent
This part represents half of the points of the end-term. In my opinion, it is reasonable to spend around 20min on Qeustion 1, 40min on Question 2 and 30min on Question 3.

\begin{exercise}[ND Derivations]
	Establish $\vdash \neg\exists x\forall y(A(x,y)\Leftrightarrow \neg A(y,y) )$ and $\vdash_c (A\to B)\lor (B\to A)$.
\end{exercise}

\vspace{0.3cm}
\begin{exercise}[F-Translation]
	We study the following variant of the Friedman translation
	\begin{align*}
		P^F &:= P \lor F & (A\lor B)^F &:= A^F\lor B^F\\
		(A\to B)^F &:= A^F\to B^F & (\forall x A)^F &:=\forall  x A^F\\
		(A\land B)^F &:=A^F\land B^F & (\exists x A)^F & := \exists x A^F
	\end{align*}
	where $F$ is a closed formula. So $A^F$ is $A$ with every atom $P$ replaced by $P\lor F$, in particular $\bot^F\Leftrightarrow F$.
	\begin{enumerate}
		\item
		Verify that $\vdash F \to A^F$ by induction on $A$.
		\item
		The proof that $\Gamma\vdash A$ implies $\Gamma^F\vdash A^F$ is by induction on $\Gamma\vdash A$. Give the case for (Exp).
		\item
		Justify briefly why $H\!A \vdash H\!A^F$, i.e.~HA proves every $F$-translated axiom.
		\item
		Given $H\!A\vdash \neg\neg \exists x.t(x)=0$ for closed $t$, choose $F$ suitable and derive that $H\!A\vdash\exists x.t(x)=0$.
	\end{enumerate}
\end{exercise}

\renewcommand{\to}{\rightarrow}

\vspace{0.3cm}
\begin{exercise}[Strong Normalisation]
	We extend STLC with a void type $\bot$
	$$\infer[BE]{\Gamma\vdash E\,t:\sigma}{\Gamma\vdash t:\bot}
	$$
	with the congruence reduction $E\,t\longrightarrow_\beta E\,t'$ whenever $t\longrightarrow_\beta t'$.
	Also, we consider terms $E\,t$ neutral.
	\begin{enumerate}
		\item
		For $\RED_\bot=\{  \Gamma\vdash t:\bot \mid t\in \SN \}$ establish:
		\begin{enumerate}
			\item[P1:]
			$t\in \RED_\bot \to t\in \SN$
			\item[P2:]
			$t\in \RED_\bot \to (t\longrightarrow_\beta t') \to t'\in \RED_\bot$
			\item[P3:]
			$t\in \NE\to (\forall t'.\,(t\longrightarrow_\beta t')\to t'\in \RED_\bot)\to t\in \RED_\bot$
		\end{enumerate}
		\item
		Extend the inductive proof that $\Gamma\vdash t:\sigma$ implies $t\{\rho\}\in \RED_\sigma$ by giving the case for (BE).
		Recall that $\rho=u_1/x_1,\dots,u_k/x_k$ substitutes all free variables $x_i$ of $t$ by $u_i\in \RED_{\sigma_i}$ where $(x_i:\sigma_i)\in \Gamma$.
		\item
		Derive that there is no closed term $t:\bot$. What does this mean for NJ (without quantifier rules)?
	\end{enumerate}
\end{exercise}


\newpage
\appendix

\section{Natural Deduction Systems}

NJ is the system $\Gamma\vdash A$ and NK is the system $\Gamma\vdash_c A$ with all rules of NJ and additionally (DN).

\begin{gather*}
	\inference[Ax]{A\in \Gamma}{\Gamma\vdash A}\hspace{2em}
	\inference[Exp]{\Gamma\vdash\bot}{\Gamma\vdash A}\hspace{2em}
	\inference[II]{\Gamma,A\vdash B}{\Gamma\vdash A\to B}\hspace{2em}
	\inference[IE]{\Gamma\vdash A\to B& \Gamma \vdash A}{\Gamma \vdash B}\\[0,5cm]
	\inference[CI]{\Gamma\vdash A&\Gamma\vdash B}{\Gamma \vdash A\land B}\hspace{2em}
	\inference[CE1]{\Gamma\vdash A\land B}{\Gamma \vdash A}\hspace{2em}
	\inference[CE2]{\Gamma\vdash A\land B}{\Gamma \vdash B}\\[0,5cm]
	\inference[DI1]{\Gamma\vdash A}{\Gamma\vdash A\lor B}\hspace{2em}
	\inference[DI2]{\Gamma\vdash B}{\Gamma\vdash A\lor B}\hspace{2em}
	\inference[OE]{\Gamma\vdash A\lor B&\Gamma,A\vdash C&\Gamma,B\vdash C}{\Gamma \vdash C}\\[0,5cm]
	\inference[AI]{\Gamma \vdash A & x\not\in FV(\Gamma)}{\Gamma\vdash \forall x A}\hspace{2em}
	\inference[AE]{\Gamma\vdash \forall x A}{\Gamma\vdash A[t/x]}\\[0,5cm]
	\inference[EI]{\Gamma\vdash A[t/x]}{\Gamma\vdash \exists x A}\hspace{2em}
	\inference[EE]{\Gamma\vdash \exists x A&\Gamma,A\vdash B&x\not\in FV(\Gamma,B)}{\Gamma\vdash B}\hspace{2em}
	\inference[DN]{\Gamma,\neg A\vdash_c\bot}{\Gamma\vdash_c A}
\end{gather*}

\vspace{0.5cm}
\section{Heyting Arithmetic (HA)}

In the language of arithmetics $(0,S,+,\cdot)$ with equality consider the following axioms
\begin{align*}
	x+0&=x & x+ (S y)&= S(x + y)\\
	x\cdot 0&= 0 & x \cdot (Sy) &= x+ x\cdot y\\
	0&\not= Sx & x &= y \text{ whenever } Sx = Sy
\end{align*}
as well as the following axiom scheme for every formula A:
$$A(0) \to (\forall x A(x) \to A (Sx)) \to \forall x A(x)$$

\vspace{0.5cm}
\section{Simply Typed Lambda Calculus}

The typing assignment of STLC is defined as follows:

$$
\infer{\Gamma \vdash x : \sigma}{(x:\sigma)\in \Gamma}\hspace{0.8cm}
\infer{\Gamma \vdash \lambda x^\sigma.t : \sigma\rightarrow \tau}{\Gamma, x:\sigma\vdash t:\tau}\hspace{0.8cm}
\infer{\Gamma\vdash tu:\tau}{\Gamma\vdash t :\sigma\rightarrow \tau & \Gamma \vdash u :\sigma}
$$
Recall that the strongly normalising terms $\SN$ are defined inductively:

$$\infer{t\in \SN}{\forall t'.\,(t\longrightarrow_\beta t') ~\to~ t'\in \SN}$$
The set $\NE$ of neutral terms contains all terms that are not abstractions.

\end{document}

