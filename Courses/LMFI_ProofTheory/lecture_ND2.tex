\documentclass[xcolor=dvipsnames,aspectratio=169,handout]{beamer}
\usecolortheme{orchid}
\setbeamertemplate{footline}[frame number]


\usepackage[utf8]{inputenc}
\usepackage[english]{babel}
\usepackage[OT1]{fontenc}
\usepackage{hyperref}
\usepackage[inference]{semantic}
\usepackage{graphicx}
\usepackage{amsfonts,amsmath}
\usepackage{cleveref}
\usepackage{scalerel}
\usepackage{multicol}
\usepackage{xspace}
\usepackage{amsmath}
\usepackage{amssymb}
\usepackage{stmaryrd}
\usepackage{proof}
\usepackage{natbib}
\usepackage{booktabs}

\usepackage{mathtools}
\usepackage{mathpartir}
\usepackage{colonequals}
\usepackage{xspace}
% Commands for correct abbreviations


\usepackage{graphicx}
\usepackage{bussproofs}
\usepackage{tikz}
\usetikzlibrary{shapes,arrows,chains}

\EnableBpAbbreviations

\usepackage{tikz}
\usepackage{tikz-cd}
\usetikzlibrary{calc}
\usetikzlibrary{arrows.meta}
\def\checkmark{\tikz\fill[scale=0.4](0,.35) -- (.25,0) -- (1,.7) -- (.25,.15) -- cycle;}

\renewcommand{\to}{\Rightarrow}
\newcommand{\TT}{\mathcal{T}}
\newcommand{\MM}{\mathcal{M}}
\newcommand{\KK}{\mathcal{K}}
\newcommand{\UU}{\mathcal U}

\usepackage{dsfont}

\title[Natural Deduction 2]{Natural Deduction 2}
\author[Dominik Kirst]{Dominik Kirst}

\date[October 14th, 2025]{LMFI Proof Theory Course\\October 14th, 2025}


% BOXES
\usepackage[many]{tcolorbox}
\newtcolorbox{emptybox}[1][]{
	beamer,
	width=(0.7\textwidth),
	% enlarge left by=-3pt,
	titlerule=0mm,
	colframe=white,
	coltitle=black,
	bottom=6pt, 
	top=-12pt,
	left=6pt,  
	right=6pt,
	notitle, 
	adjusted title={},
	outer arc=.5mm,
	arc=.5mm,
	no shadow,
	fuzzy shadow={1mm}{-1mm}{-1.2mm}{.7mm}{black!20},
	interior titled code={}
}

\newtcolorbox{widerbox}[1][]{
	beamer,
	width=(0.8\textwidth),
	% enlarge left by=-3pt,
	titlerule=0mm,
	colframe=white,
	coltitle=black,
	bottom=6pt, 
	top=-12pt,
	left=6pt,  
	right=6pt,
	notitle, 
	adjusted title={},
	outer arc=.5mm,
	arc=.5mm,
	no shadow,
	fuzzy shadow={1mm}{-1mm}{-1.2mm}{.7mm}{black!20},
	interior titled code={}
}

\newtcolorbox{widebox}[1][]{
	beamer,
	width=(\textwidth),
	% enlarge left by=-3pt,
	titlerule=0mm,
	colframe=white,
	coltitle=black,
	bottom=6pt, 
	top=-12pt,
	left=6pt,  
	right=6pt,
	notitle, 
	adjusted title={},
	outer arc=.5mm,
	arc=.5mm,
	no shadow,
	fuzzy shadow={1mm}{-1mm}{-1.2mm}{.7mm}{black!20},
	interior titled code={}
}


% TIMELINE
\usepackage{tikz}
\usepackage{graphicx}
\usepackage{stackengine}
\usetikzlibrary{trees, shapes, calc, positioning, shadows}

\definecolor{darkcerulean}{rgb}{0.063661, 0.257392, 0.477463}
\definecolor{scooter}{rgb}{0.161162, 0.775760, 0.885416}


% #1 – box name
% #2 – left color
% #3 – right color
% #4 – Title (may contain #1)
\newcommand{\newlargebox}[4]{
	\newtcolorbox{#1}[1][]{
		beamer,
		width=\textwidth+7pt,
		enlarge left by=-3pt,
		titlerule=3mm,
		colframe=white,
		coltitle={#2},
		bottom=2pt, 
		top=-4pt,
		left=6pt,  
		toptitle=2pt,
		bottomtitle=-2pt,
		fonttitle=\bfseries\large,
		adjusted title={#4},
		outer arc=.5mm,
		arc=.5mm,
		no shadow,
		fuzzy shadow={1mm}{-1mm}{-1.2mm}{.7mm}{black!20},
		interior titled code={
			\path [left color = {#2}, right color = {#3}]
			(title.south west) + (8pt, 0) rectangle ++(\textwidth-1pt, 0.02);
		}
	}
}


\newlargebox{defaultbox}{darkcerulean}{scooter}{#1} 

\usepackage{fancyvrb}
\usepackage{xcolor}


\begin{document}

%\beamerdefaultoverlayspecification{<+->}
\definecolor{red}{RGB}{204,0,0}
\definecolor{yellow}{RGB}{228,242,31}
\definecolor{green}{RGB}{0,204,0}

\newcommand\refs[1]{%
	\begin{textblock*}{8cm}(0.3cm,9.2cm)%
		\scriptsize {\color{gray}#1}
	\end{textblock*}
}

\begin{frame}
	\maketitle
\end{frame}


\begin{frame}{Recap}
	Last week, we studied natural deduction systems:
	\begin{itemize}
		\vspace{0.3cm}
		\item
		Intuitionistic, classical and minimal versions
		\vspace{0.3cm}
		\item
		Structural properties like admissible weakening: $\Gamma\vdash A$ with $\Gamma\subseteq \Delta$ implies $\Delta \vdash A$
		\vspace{0.3cm}
		\item
		Equivalence to sequent calculus: $\Gamma\vdash \Xi$ implies $\Gamma \vdash \bigvee \Xi$
		\vspace{0.3cm}
		\item
		Double-negation translation: $\Gamma\vdash_c A$ if and only if $\Gamma^*\vdash A^*$
	\end{itemize}
	\vspace{0.3cm}
	All these results were syntactic, today we take a more semantic perspective!
\end{frame}

\begin{frame}{Plan for Today}
	\begin{itemize}
		\item 
		Henkin's completeness proof:
		\begin{itemize}
			\item
			Sketch of the usual proof
			\item
			Constructive refinement
		\end{itemize}
		\vspace{0.3cm}
		\item
		Kripke semantics:
		\begin{itemize}
			\item
			Version for the full syntax
			\item
			Version for the negative fragment
		\end{itemize}
		\vspace{0.3cm}
		\item
		Kripke completeness:
		\begin{itemize}
			\item
			Sketch of the usual proof
			\item
			Semantic cut-elimination
		\end{itemize}
	\end{itemize}
	
	\vspace{0.5cm}
	Mostly based on this paper:\\ \scriptsize \url{https://www.ps.uni-saarland.de/Publications/documents/ForsterEtAl_2021_Completeness.pdf}
\end{frame}

\begin{frame}{Henkin Completeness Proof: Blueprint}
	\begin{theorem}
		For every sentence $A$, if $\vDash A$ then $\vdash_c A$.
	\end{theorem}
	\pause
	\begin{proof}[Outline]
		\begin{enumerate}
			\pause
			\item
			Assume $\vDash A$ but suppose $\not\vdash_c A$, so in particular $\neg A \not\vdash_c \bot$.
			\pause
			\item
			Extend $\{\neg A\}$ into maximal complete theory $\TT$ (Lindenbaum Lemma).
			\pause
			\item
			Give $\TT$ the structure of a syntactic model $\MM_\TT$.
			\pause
			\item
			Establish that $B\in \TT$ exactly if $\MM_\TT\vDash B$ (Truth Lemma).
			\pause
			\item
			Conclude that the contradiction that $\{\neg A\}$ has a model (Model Existence).
			\qedhere
		\end{enumerate}
	\end{proof}
	
	\pause
	We go through these steps in detail, focusing on the negative $(\bot,\to,\forall)$-fragment.
\end{frame}

\begin{frame}{Henkin Completeness Proof: Model Existence}
	\begin{lemma}[Lindenbaum]
		For every consistent set $\Gamma$ of sentences there is a consistent theory $\TT\supseteq \Gamma$ with:
		\begin{itemize}
			\item
			$A\to B\in \TT$ if and only if $A\in \TT$ implies $B\in \TT$
			\item
			$\forall x A\in \TT$ if and only if for all terms $t$ we have $A[t/x]\in \TT$
		\end{itemize}
	\end{lemma}
	
	\pause
	\begin{theorem}
		For every consistent set $\Gamma$ of sentences there is a model $\MM$ with $\MM\vDash \Gamma$.
	\end{theorem}
	\pause
	\vspace{-0.2cm}
	\begin{proof}
		\begin{enumerate}
			\pause
			\item
			Use the Lindenbaum Lemma to obtain $\TT\supseteq \Gamma$ as above.
			\pause
			\item
			Define the model $\MM_\TT$ over the domain of (open) terms by:
			$$f^{\MM_\TT}(t_1,\dots ,t_k):=f(t_1,\dots ,t_k)
			\hspace{3em}
			P^{\MM_\TT}(t_1,\dots ,t_k):=P(t_1,\dots ,t_k)\in \TT
			$$
			\item
			\pause
			Establish $A[\sigma]\in \TT$ exactly if $\MM_\TT\vDash_\sigma A$, so in particular $\MM_\TT\vDash \Gamma$.
			\qedhere
		\end{enumerate}
	\end{proof}
\end{frame}

\begin{frame}{Henkin Completeness Proof: Lindenbaum Lemma}
	\begin{lemma}[Lindenbaum]
		For every consistent set $\Gamma$ of sentences there is a consistent theory $\TT\supseteq \Gamma$ with:
		\begin{itemize}
			\item
			$A\to B\in \TT$ if and only if $A\in \TT$ implies $B\in \TT$
			\item
			$\forall x A\in \TT$ if and only if for all terms $t$ we have $A[t/x]\in \TT$
		\end{itemize}
	\end{lemma}
	
	\pause
	\begin{proof}
		\pause
		Using an enumeration of formulas such that $x_n$ is fresh for all $A_m$ with $m\le n$ we define:
		$$\textstyle \TT_0 := \Gamma \cup \{A(x_n){\to} \forall xA_n(x)\mid n\in \mathbb{N}\}
		\hspace{1.5em}
		\TT_{n+1}:= \TT_n\cup \{A_n \mid \TT_n,A\not\vdash\bot \}
		\hspace{1.5em}
		\TT :=\bigcup_{n\in \mathbb{N}} \TT_n
		$$
		\vspace{-0.5cm}
		\begin{enumerate}
			\pause
			\item
			Adding Henkin axioms for $\TT_0$ maintains consistency by use of the Drinker Paradox.
			\item
			Each iteration step for $\TT_{n+1}$ maintains consistency.
			\pause
			\item
			The limit $\TT$ is maximal and thus deductively closed.
			\pause
			\item
			The wished properties follow by deductive closure and the Henkin axioms.
			\qedhere
		\end{enumerate}
	\end{proof}
\end{frame}

\begin{frame}{Henkin Completeness Proof: Minimal Refinement}
	Note that for the completeness result all steps but the very first were constructive!
	\pause
	
	If we now further restrict to the $(\to,\forall)$-fragment, something interesting happens:
	\pause
	\begin{lemma}[Lindenbaum]
		For every set $\Gamma$ of sentences and every sentence $A_\bot$ there is a theory $\TT\supseteq \Gamma_\bot$ with:
		\begin{itemize}
			\item
			$A_\bot \in \TT$ if and only if $\Gamma\vdash_c A_\bot$
			\item
			$A\to B\in \TT$ if and only if $A\in \TT$ implies $B\in \TT$
			\item
			$\forall x A\in \TT$ if and only if for all terms $t$ we have $A[t/x]\in \TT$
		\end{itemize}
	\end{lemma}
		
		\pause
		\begin{proof}
			\pause
			Similar as before, extending $\textstyle \TT_0 := \Gamma\cup \{A_\bot{\to} A_n\mid n\in \mathbb N \} \cup \{A(x_n){\to} \forall xA_n(x)\mid n\in \mathbb{N}\}$.\\
			For consistency, note that $\Gamma, A_\bot{\to} A_n\vdash_c A_\bot$ implies $\Gamma\vdash_c A_\bot$ (Peirce's Law).
		\end{proof}
		
	\pause
	\vspace{0.3cm}
	So the induced model $\MM_\TT$ of $\Gamma$ satisfies that $\MM_\TT\vDash A_\bot$ implies $\Gamma\vdash_c A_\bot$.\\
	\pause
	$\Rightarrow$ From this, the completeness theorem follows constructively!
\end{frame}

\begin{frame}{Henkin Completeness Proof: Constructive Refinement}
	The previous proof does not extend to the $(\bot,\to,\forall)$-fragment:
	If $A_\bot$ is provable, then $\TT$ contains all formulas, including contradictions and hence doesn't induce a model.
	
	\vspace{0.3cm}
	\pause
	However, we can relax the notion of model to accommodate this additional case:
	\pause
	\begin{definition}
		We relativise the satisfaction relation to an arbitrary proposition $P$ via $\MM \vDash^P_\alpha \bot:= P$ and call $\MM$ exploding if for some choice of $P$ we have $\MM\vDash^P_\alpha \bot \to A$ for all $A$.
	\end{definition}
	
	\vspace{0.3cm}
	\pause
	Then completeness can be derived similar to the minimal case:
	\begin{theorem}
		For every sentence $A$, if $A$ is valid in every exploding model then $\vdash_c A$.
	\end{theorem}
	\pause
	\begin{proof}
		Obtain $\TT$ as before and observe that $\MM_\TT$ is exploding for the choice $P:=(\vdash_c A)$.
	\end{proof}
\end{frame}

\begin{frame}{Kripke Semantics}
	\pause
	A standard semantics of intuitionistic logic is given by Kripke models:
	\pause
	\begin{definition}
		A Kripke model $\KK=(W,\preceq, (M_w)_{w\in W})$ consists of the following data:
		\begin{itemize}
			\item $(W,\preceq)$ is a pre-order, so reflexive and transitive.
			\item $(M_w)_{w\in W}$ is a family of models such that whenever $w\preceq w'$ and $x_1,...,x_k\in M_w$:
			$$\hspace{-0.8cm}x_1,...,x_k\in M_{w'}
			\hspace{1em}
			f^{M_{w}}(x_1,...,x_k)=f^{M_{w'}}(x_1,...,x_k)
			\hspace{1em}
			P^{M_{w}}(x_1,...,x_k)\rightarrow P^{M_{w'}}(x_1,...,x_k)
			$$
		\end{itemize}
		\pause
		We define the forcing relation $\KK\Vdash^w_\alpha A$ mostly as in $M_w\vDash_\alpha A$ but with:
		\begin{align*}
			\KK\Vdash^w_\alpha A\to B&\text{ iff for for all $w'\succeq w$ we have that $\KK\Vdash^{w'}_\alpha A$ implies $\KK\Vdash^{w'}_\alpha B$}\\
			\KK\Vdash^w_\alpha \forall x A&\text{ iff for for all $w'\succeq w$ and $a\in M_{w'}$ we have that $\KK\Vdash^{w'}_{\alpha[x:=a]} A$}
		\end{align*}
	\end{definition}
\end{frame}

\begin{frame}{Kripke Semantics: Soundness}
	\pause
	\begin{lemma}[Persistence]
		If $\KK\Vdash^w_\alpha A$ and $w'\succeq w$, then also $\KK\Vdash^{w'}_\alpha A$.
	\end{lemma}
	\vspace{-0.2cm}
	\pause
	\begin{theorem}
		If $\Gamma\vdash A$ then $\KK\Vdash^{w}_\alpha \Gamma$ implies $\KK\Vdash^{w}_\alpha A$ for all $w$ and $\alpha$.
	\end{theorem}
	\vspace{-0.2cm}
	\pause
	\begin{proof}
		By induction on the derivation of $\Gamma\vdash A$, here are some interesting cases:
		\pause
		\begin{itemize}
			\pause
			\item
			For the rule II, we assume that $\KK\Vdash^{w}_\alpha \Gamma$ and show $\KK\Vdash^{w}_\alpha A \to B$, so that for every $w'\succeq w$ we have that $\KK\Vdash^{w'}_\alpha A$ implies $\KK\Vdash^{w'}_\alpha B$.
			By persistence we hence have $\KK\Vdash^{w'}_\alpha \Gamma,A$ and conclude $\KK\Vdash^{w'}_\alpha B$ by the inductive hypothesis for $w:=w'$.
			\pause
			\item
			For the rule AI, we assume that $\KK\Vdash^{w}_\alpha \Gamma$ and show $\KK\Vdash^{w}_\alpha \forall x A$, so that for every $w'\succeq w$ and $a\in M_{w'}$ we have that $\KK\Vdash^{w'}_{\alpha[x:=a]} A$.
			Using persistence and $x\not\in FV(\Gamma)$, the first assumption also yields $\KK\Vdash^{w'}_{\alpha[x:=a]} \Gamma$, from which we conclude $\KK\Vdash^{w'}_{\alpha[x:=a]} A$ by the inductive hypothesis for $w:=w'$ and $\alpha := \alpha[x:=a]$.
			\qedhere
		\end{itemize}
	\end{proof}
\end{frame}

\begin{frame}[fragile]{Kripke Semantics: Applications}
	\pause
	We can give a simple Kripke models to argue non-derivabilities:
	
	\pause
	\vspace{0.2cm}
	$$
	\begin{tikzcd}[row sep=huge]
		w_2 \Vdash P &
		 &
		w_3 \not\Vdash P
		\\
		& w_1\not\Vdash P\arrow[ul]\arrow[ur] &
	\end{tikzcd}
	$$
	\vspace{0.2cm}
	
	\pause
	\begin{itemize}
		\item
		For a language with a propositional constant $P$ consider the model depicted
		\pause
		\item
		We have $w_2 \Vdash P$ and $w_3\Vdash \neg P$
		\pause
		\item
		However, we have neither $w_1 \Vdash P$ nor $w_1\Vdash \neg P$, so $w_1\not\Vdash P\lor \neg P$
		\pause
		\item
		Thus $\not\vdash P\lor \neg P$ because otherwise $w_1\not\Vdash P\lor \neg P$ by soundness
	\end{itemize}
\end{frame}

\begin{frame}{Kripke Semantics: Negative Fragment}
	\pause
	To study completeness, we again restrict to the negative $(\bot,\to,\forall)$-fragment.\\
	\pause
	This allows us to simplify the semantics in the following way:
	\vspace{0.3cm}
	\pause
	\begin{definition}
		A constant domain Kripke model $\KK=(W,\preceq, D,(M_w)_{w\in W})$ consists of the following:
		\begin{itemize}
			\item $(W,\preceq)$ is a pre-order, so reflexive and transitive.
			\item $(M_w)_{w\in W}$ is a family of $D$-models such that whenever $w\preceq w'$ and $x_1,...,x_k\in D$:
			$$
			f^{M_{w}}(x_1,...,x_k)=f^{M_{w'}}(x_1,...,x_k)
			\hspace{3em}
			P^{M_{w}}(x_1,...,x_k)\rightarrow P^{M_{w'}}(x_1,...,x_k)
			$$
		\end{itemize}
		\pause
		We define the forcing relation $\KK\Vdash^w_\alpha A$ mostly as before but with:
		\begin{align*}
			\KK\Vdash^w_\alpha \forall x A&\text{ iff for all $a\in D$ we have that $\KK\Vdash^{w}_{\alpha[x:=a]} A$}
		\end{align*}
	\end{definition}
\end{frame}

\begin{frame}{Kripke Completeness: Blueprint}
	\begin{theorem}
		For every sentence $A$, if $\Vdash A$ then $\vdash A$.
	\end{theorem}
	\begin{proof}[Outline]
		\begin{enumerate}
			\pause
			\item
			Assume $\vDash A$ but set out to prove $\vdash A$.
			\pause
			\item
			Consider the pre-order of consistent contexts ordered by inclusion.
			\pause
			\item
			Form the universal model $\UU$ over the constant domain of terms by:
			$$f^{M_{\Gamma}}(x_1,...,x_k)=f(x_1,...,x_k)\hspace{3em}P^{M_{\Gamma}}(x_1,...,x_k):= \Gamma\vdash P(x_1,...,x_k)$$
			\item
			\pause
			Observe that $\UU\Vdash^\Gamma B$ exactly if $\Gamma \vdash B$ (Truth Lemma).
			\pause
			\item
			By assumption of $\vDash A$ in particular $\UU\Vdash^\emptyset_\sigma A$, so $\vdash A$ follows.
			\qedhere
		\end{enumerate}
	\end{proof}
\end{frame}

\begin{frame}{Kripke Completeness: Truth Lemma}
	\begin{lemma}[Truth Lemma]
		In the universal model, we have $\UU\Vdash^\Gamma_\sigma A$ exactly if $\Gamma \vdash A[\sigma]$.
	\end{lemma}
	\pause
	\begin{proof}
		\pause
		By induction on $A$, the only really interesting case is $A\to B$ from left to right:
		\pause
		\begin{enumerate}
			\item
			By IH, we can assume that $\Delta \vdash A[\sigma]$ implies $\Delta\vdash B[\sigma]$ for all consistent $\Delta\supseteq \Gamma$.
			\item
			To show $\Gamma \vdash A[\sigma]\to B[\sigma]$, it suffices to show $\Gamma,A[\sigma]\vdash B[\sigma]$.
			\begin{enumerate}[a.]
				\item
				If $\Gamma,A[\sigma]$ is consistent, conclude $\Gamma,A[\sigma]\vdash B[\sigma]$ by the assumption in (1).
				\item
				If $\Gamma,A[\sigma]$ is consistent, conclude $\Gamma,A[\sigma]\vdash B[\sigma]$ by the explosion rule.
				\qedhere
			\end{enumerate}
		\end{enumerate}
	\end{proof}
	\pause
	Note that we used classical logic to test whether extension maintains consistency...
\end{frame}

\begin{frame}{Kripke Completeness: Semantic Cut-Elimination}
	\pause
	As in the Henkin completeness proof, we can drop the consistency requirement by relaxing to exploding models, allowing a universal model $\UU$ over all contexts:
	\begin{itemize}
		\pause
		\item
		Then the truth lemma is constructive as no consistency checks are necessary.
		\pause
		\item
		We then can even do more, writing $\Gamma\vdash_{cf} A$ for cut-free sequent derivations:
		$$\UU\Vdash^\Gamma_\sigma A \text{ implies } \Gamma\vdash_{cf}A[\sigma]$$
		can be proven if we simultaneously prove:
		$$\text{If for all $\Delta\supseteq \Gamma$ and $B$ with $\Delta,A[\sigma]\vdash_{cf} B$ we have $\Delta\vdash_{cf} B$, then $\UU\Vdash^\Gamma_\sigma A$} $$
		\item
		\pause
		So by composition with soundness we in particular obtain $\vdash_{cf} A$ from $\vdash A$, so we eliminated cuts and since the proof is constructive it bears an actual algorithm!
	\end{itemize}
\end{frame}





\end{document}

%\begin{frame}{Bibliography}
%%\begin{frame}{Bibliography}
%\footnotesize
%\bibliographystyle{apalike}
%\bibliography{../biblio}
%\end{frame}

\appendix









\end{document}
