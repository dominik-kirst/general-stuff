\documentclass[xcolor=dvipsnames,aspectratio=169,handout]{beamer}
\usecolortheme{orchid}
\setbeamertemplate{footline}[frame number]


\usepackage[utf8]{inputenc}
\usepackage[english]{babel}
\usepackage[OT1]{fontenc}
\usepackage{hyperref}
\usepackage[inference]{semantic}
\usepackage{graphicx}
\usepackage{amsfonts,amsmath}
\usepackage{cleveref}
\usepackage{scalerel}
\usepackage{multicol}
\usepackage{xspace}
\usepackage{amsmath}
\usepackage{amssymb}
\usepackage{stmaryrd}
\usepackage{proof}
\usepackage{natbib}
\usepackage{booktabs}

\usepackage{mathtools}
\usepackage{mathpartir}
\usepackage{colonequals}
\usepackage{xspace}
\usepackage{tabularx}
% Commands for correct abbreviations


\usepackage{graphicx}
\usepackage{bussproofs}
\usepackage{tikz}
\usetikzlibrary{shapes,arrows,chains}

\EnableBpAbbreviations

\usepackage{tikz}
\usepackage{tikz-cd}
\usetikzlibrary{calc}
\usetikzlibrary{arrows.meta}
\def\checkmark{\tikz\fill[scale=0.4](0,.35) -- (.25,0) -- (1,.7) -- (.25,.15) -- cycle;}

\usepackage{dsfont}

\title[Simply Typed Lambda Calculus]{Simply Typed Lambda Calculus}
\author[Dominik Kirst]{Dominik Kirst}

\date[November 18th, 2025]{LMFI Proof Theory Course\\November 18th, 2025}


% BOXES
\usepackage[many]{tcolorbox}
\newtcolorbox{emptybox}[1][]{
	beamer,
	width=(0.7\textwidth),
	% enlarge left by=-3pt,
	titlerule=0mm,
	colframe=white,
	coltitle=black,
	bottom=6pt, 
	top=-12pt,
	left=6pt,  
	right=6pt,
	notitle, 
	adjusted title={},
	outer arc=.5mm,
	arc=.5mm,
	no shadow,
	fuzzy shadow={1mm}{-1mm}{-1.2mm}{.7mm}{black!20},
	interior titled code={}
}

\newtcolorbox{widerbox}[1][]{
	beamer,
	width=(0.8\textwidth),
	% enlarge left by=-3pt,
	titlerule=0mm,
	colframe=white,
	coltitle=black,
	bottom=6pt, 
	top=-12pt,
	left=6pt,  
	right=6pt,
	notitle, 
	adjusted title={},
	outer arc=.5mm,
	arc=.5mm,
	no shadow,
	fuzzy shadow={1mm}{-1mm}{-1.2mm}{.7mm}{black!20},
	interior titled code={}
}

\newtcolorbox{widebox}[1][]{
	beamer,
	width=(\textwidth),
	% enlarge left by=-3pt,
	titlerule=0mm,
	colframe=white,
	coltitle=black,
	bottom=6pt, 
	top=-12pt,
	left=6pt,  
	right=6pt,
	notitle, 
	adjusted title={},
	outer arc=.5mm,
	arc=.5mm,
	no shadow,
	fuzzy shadow={1mm}{-1mm}{-1.2mm}{.7mm}{black!20},
	interior titled code={}
}


% TIMELINE
\usepackage{tikz}
\usepackage{graphicx}
\usepackage{stackengine}
\usetikzlibrary{trees, shapes, calc, positioning, shadows}

\definecolor{darkcerulean}{rgb}{0.063661, 0.257392, 0.477463}
\definecolor{scooter}{rgb}{0.161162, 0.775760, 0.885416}


% #1 – box name
% #2 – left color
% #3 – right color
% #4 – Title (may contain #1)
\newcommand{\newlargebox}[4]{
	\newtcolorbox{#1}[1][]{
		beamer,
		width=\textwidth+7pt,
		enlarge left by=-3pt,
		titlerule=3mm,
		colframe=white,
		coltitle={#2},
		bottom=2pt, 
		top=-4pt,
		left=6pt,  
		toptitle=2pt,
		bottomtitle=-2pt,
		fonttitle=\bfseries\large,
		adjusted title={#4},
		outer arc=.5mm,
		arc=.5mm,
		no shadow,
		fuzzy shadow={1mm}{-1mm}{-1.2mm}{.7mm}{black!20},
		interior titled code={
			\path [left color = {#2}, right color = {#3}]
			(title.south west) + (8pt, 0) rectangle ++(\textwidth-1pt, 0.02);
		}
	}
}


\newlargebox{defaultbox}{darkcerulean}{scooter}{#1} 

\usepackage{fancyvrb}
\usepackage{xcolor}


\begin{document}

%\beamerdefaultoverlayspecification{<+->}
\definecolor{red}{RGB}{204,0,0}
\definecolor{yellow}{RGB}{228,242,31}
\definecolor{green}{RGB}{0,204,0}

\newcommand\refs[1]{%
	\begin{textblock*}{8cm}(0.3cm,9.2cm)%
		\scriptsize {\color{gray}#1}
	\end{textblock*}
}

\renewcommand{\to}{\rightarrow}

\begin{frame}
	\maketitle
\end{frame}

\begin{frame}{Slogans}
	\begin{itemize}
		\pause
		\item
		Imposes additional structure on computations
		\pause
		\vspace{0.5cm}
		\item
		Restricts to terminating computations
		\pause
		\vspace{0.5cm}
		\item
		Logical interpretation by Curry-Howard isomorphism
		\pause
		\vspace{0.5cm}
		\item
		Core of functional programming languages
		\pause
		\vspace{0.5cm}
		\item
		Extends to dependent type theories underlying proof assistants
	\end{itemize}
\end{frame}

\begin{frame}{Overview}
	Lecture 1:
	\begin{itemize}
		\item
		Simply typed lambda calculus
		\item
		Structural properties
		\item
		Normalisation
		\item
		Curry-Howard isomorphism
	\end{itemize}
	
	\vspace{0.4cm}
	Lecture 2:
	\begin{itemize}
		\item
		Strong normalisation
		\item
		Extension with products
	\end{itemize}
	
	\vspace{0.4cm}
	Lecture 3:
	\begin{itemize}
		\item
		Extension with sums
		\item
		Intersection types
	\end{itemize}
\end{frame}

\begin{frame}{Simple Types}
	\pause
	Assuming some base types $A,B,C,\dots$, we consider (simple) types:
	\begin{itemize}
		\item
		Every base type is a type.
		\item
		If $\sigma$ and $\tau$ are types, then so is $\sigma\rightarrow \tau$.
	\end{itemize}
	
	\pause
	\vspace{0.8cm}
	In STLC, terms $t$ are assigned types $\sigma$ in contexts $\Gamma$, written $\Gamma\vdash t:\sigma$, as follows:
	$$
	\infer{\Gamma \vdash x : \sigma}{(x:\sigma)\in \Gamma}\hspace{0.8cm}
	\infer{\Gamma \vdash \lambda x^\sigma.t : \sigma\rightarrow \tau}{\Gamma, x:\sigma\vdash t:\tau}\hspace{0.8cm}
	\infer{\Gamma\vdash tu:\tau}{\Gamma\vdash t :\sigma\rightarrow \tau & \Gamma \vdash u :\sigma}
	$$
	\pause
	In this presentation, variables $x$ in abstractions $\lambda x^\sigma.t$ are usually annotated with types.
\end{frame}

\begin{frame}{Example}
	$$\infer{\vdash \lambda x y z. x z(yz) : (A\to B\to C)\to (A\to B) \to A \to C}{
		\infer{(x:A\to B\to C),(y:A \to B),(z:A)\vdash x z(yz):C}{
			\infer{\Gamma\vdash xz: B\to C }{\infer{\Gamma\vdash x:A\to B\to C}{}&\infer{\Gamma\vdash z:A}{}}
			&
			\infer{\Gamma\vdash yz: B}{\infer{\Gamma\vdash y : A\to B}{}&\infer{\Gamma\vdash z: A}{}}
		}
	}$$
\end{frame}

\begin{frame}{Type Checking, Type Inference and Type Inhabitation}
	\pause
	We distinguish three different decision problems related to typing:
	\begin{enumerate}
		\pause
		\item
		\vspace{0.3cm}
		Type checking: given $\Gamma$, $t$ and $\sigma$, does $\Gamma\vdash t:\sigma$ hold?\\
		$\Rightarrow$ Decidable since the typing rules are deterministic!
		\pause
		\item
		\vspace{0.3cm}
		Type inference: given $\Gamma$ and $t$, is there $\sigma$ such that $\Gamma\vdash t:\sigma$ holds?\\
		$\Rightarrow$ Decidable since most general types can be derived!
		\vspace{0.3cm}
		\pause
		\item
		Type inhabitation: given $\sigma$, are there $t$ and $\Gamma$ such that $\Gamma\vdash t:\sigma$ holds?\\
		$\Rightarrow$ Decidable, but not trivially so!
	\end{enumerate}
	
	\pause
	\vspace{0.3cm}
	In extensions of STLC, these may well become undecidable...
\end{frame}

\begin{frame}{Weakening and Substitutivity}
	\pause
	\begin{fact}[Weakening]
		If $\Gamma\vdash t : \sigma$ and $\Gamma\subseteq \Delta$, then $\Delta\vdash t : \sigma$
	\end{fact}
	\pause
	\begin{proof}
		By induction on $\Gamma\vdash t : \sigma$ with $\Delta$ generalised.
	\end{proof}
	
	\pause
	\vspace{0.8cm}
	\begin{fact}[Substitutivity]
		If $\Gamma, x:\tau \vdash t : \sigma$ and $\Gamma \vdash u :\tau$, then $\Gamma\vdash t\{u/x\}:\sigma$.
	\end{fact}
	\pause
	\begin{proof}
		By induction on $\Gamma, x:\tau \vdash t : \sigma$.
	\end{proof}
\end{frame}

\begin{frame}{Subject Reduction}
	\pause
	\begin{fact}
		If $\Gamma\vdash t : \sigma$ and $t\longrightarrow_\beta u$, then $\Gamma\vdash u: \sigma$.
	\end{fact}
	\pause
	\begin{proof}
		\pause
		By induction on $t\longrightarrow_\beta u$ with $\Gamma$ generalised:
		\pause
		\begin{itemize}
			\item
			If $(\lambda x^\sigma.t)u\longrightarrow_\beta t\{u/x\}$ and $\Gamma \vdash(\lambda x^\sigma.t)u:\tau$, then we know that $\Gamma, x:\sigma \vdash t :\tau $ and $\Gamma\vdash u :\sigma$.
			Then the claim $\Gamma\vdash t\{u/x\}:\tau$ follows by substitutivity.
			\pause
			\item
			If $(\lambda x^\sigma. t) \longrightarrow_\beta (\lambda x^\sigma.t')$ via $t\longrightarrow_\beta t'$ and $\Gamma \vdash (\lambda x^\sigma. t):\sigma\to\tau$, then we know that $\Gamma, x:\sigma\vdash t:\tau$.
			Then by the IH for $\Gamma, x:\sigma$ we obtain $\Gamma, x:\sigma\vdash t':\tau$ and thus $\Gamma \vdash (\lambda x^\sigma. t'):\sigma\to\tau$ as desired.
			\pause
			\item
			If $t u \longrightarrow_\beta t' u$ via $t \longrightarrow_\beta t'$ and $\Gamma\vdash tu:\tau$, then we know that $\Gamma \vdash t:\sigma\to\tau$ and $\Gamma\vdash u :\sigma$.
			Then by the IH also $\Gamma \vdash t':\sigma\to\tau$ and thus $\Gamma\vdash t'u:\tau$ as desired.
			\pause
			\item
			If $t u \longrightarrow_\beta t' u$ via $u \longrightarrow_\beta u'$ and $\Gamma\vdash tu:\tau$, then we know that $\Gamma \vdash t:\sigma\to\tau$ and $\Gamma\vdash u :\sigma$.
			Then by the IH also $\Gamma \vdash u':\sigma$ and thus $\Gamma\vdash tu':\tau$ as desired.
			\qedhere
		\end{itemize}
	\end{proof}
\end{frame}

\begin{frame}{Subject Expansion}
	Is typing also preserved by expansion?
	
	\vspace{0.3cm}
	\pause
	If $\Gamma\vdash u : \sigma$ and $t\longrightarrow_\beta u$, then $\Gamma\vdash t : \sigma$?
	
	\vspace{0.8cm}
	\pause
	No!
	\pause
	$$(\lambda x y.x)(\lambda x.x)\Omega~ \longrightarrow_\beta^*~ \lambda x.x$$
	\pause
	Here $\lambda x.x$ on the left can be assigned any type $\sigma\to\sigma$ while $\Omega$ is untypable.
\end{frame}

\begin{frame}{Weak Normalisation: Statement}
	For STLC, there are two notions of normalisation:
	\begin{itemize}
		\pause
		\item
		\emph{Weak normalisation} states that every well-typed term has a normal form.
		\pause
		\item
		\emph{Strong normalisation} states that every reduction sequence ends in a normal form.
	\end{itemize}
	\pause
	The latter implies the former but the converse is not true in general...
	
	\pause
	\vspace{0.5cm}
	In this lecture, we sketch a simple argument for the weaker property:
	\begin{theorem}[Weak Normalisation]
		If $\Gamma\vdash t : \sigma$, then $t$ has a normal form.
	\end{theorem}
	\vspace{0.5cm}
	\pause
	Does the converse also hold?
	\pause
	No, because $xx$ is a normal form but untypable!
\end{frame}

\begin{frame}{Weak Normalisation: Proof Sketch}
	\begin{theorem}[Weak Normalisation]
		If $\Gamma\vdash t : \sigma$, then $t$ has a normal form.
	\end{theorem}
	\pause
	\begin{proof}[Proof Sketch]
		\pause
		We can define a well-order $o(t)$ such that whenever $t$ is not yet normal, there is $t'$ with $t\longrightarrow_\beta t'$ and $o(t')<o(t)$.
		Then the claim follows by well-foundedness of $o$.
		\begin{itemize}
			\pause
			\item
			The degree of a type is its nesting depth of arrows.
			\pause
			\item
			The degree of a redex is the degree of its arrow type.
			\pause
			\item
			The degree $d(t)$ of a term $t$ is the max degree of its redexes.
			\pause
			\item
			Then $o(t)=(d(t),n)$ where and $n$ is the number of redexes of max degree.
		\end{itemize}
		\pause
		So if $t$ is not yet normal, just reduce one of it's redexes of max degree to obtain $t'$.
	\end{proof}
	\pause
	The proof of strong normalisation that we discuss next week will be more intricate...
\end{frame}

\begin{frame}{The Magic Trick}
	Reconsider the typing example from the beginning:
	\vspace{0.7cm}
	
	$$\infer{\vdash \lambda x y z. x z(yz) : (A\to B\to C)\to (A\to B) \to A \to C}{
	\infer{(x:A\to B\to C),(y:A \to B),(z:A)\vdash x z(yz):C}{
	\infer{\Gamma\vdash xz: B\to C }{\infer{\Gamma\vdash x:A\to B\to C}{}&\infer{\Gamma\vdash z:A}{}}
	&
	\infer{\Gamma\vdash yz: B}{\infer{\Gamma\vdash y : A\to B}{}&\infer{\Gamma\vdash z: A}{}}
	}
	}$$
	
	\vspace{0.7cm}
	What happens if we delete all term information?
\end{frame}

\begin{frame}{The Magic Trick}
	Reconsider the typing example from the beginning:
	\vspace{0.7cm}
	
	$$\infer{\vdash  (A\to B\to C)\to (A\to B) \to A \to C}{
		\infer{(A\to B\to C,A \to B,A)\vdash C}{
			\infer{\Gamma\vdash B\to C }{\infer{\Gamma\vdash A\to B\to C}{}&\infer{\Gamma\vdash A}{}}
			&
			\infer{\Gamma\vdash B}{\infer{\Gamma\vdash A\to B}{}&\infer{\Gamma\vdash  A}{}}
		}
	}$$
	
	\vspace{0.7cm}
	We obtain a proof in (intuitionistic) natural deduction!
\end{frame}

\begin{frame}{The Curry-Howard Isomorphism}
	We can interpret logical propositions as computational types!
	\vspace{0.2cm}
	\pause
	\begin{itemize}
		\item
		Proof checking becomes type checking
		\vspace{0.2cm}
		\pause
		\item
		Provability becomes type inhabitation
		\vspace{0.2cm}
		\pause
		\item
		Cut-elimination becomes normalisation
	\end{itemize}
	
	\vspace{0.7cm}
	\pause
	This isomorphism is most direct for intuitionistic logic but can also be extended to accommodate classical logic via well-known program transformations.
\end{frame}

\begin{frame}{Perspective}
	This interpretation goes a long way:
	
	\vspace{0.5cm}
	\begin{center}
		\begin{tabularx}{.628\textwidth}{|c|c|}
			\hline
			Implication& Function types\\
			\hline
			Conjunction & Product types\\
			\hline
			Disjunction & Sum types\\
			\hline
			Truth & Unit type\\
			\hline
			Falsity & Void type\\
			\hline
			Universal quantification & Dependent product type\\
			\hline
			Existential quantification & Dependent sum type\\
			\hline
		\end{tabularx}
	\end{center}
	
	\pause
	\vspace{0.5cm}
	This also tells us what the respective proofs are and how they compute!
\end{frame}




\end{document}



%\begin{frame}{Bibliography}
%%\begin{frame}{Bibliography}
%\footnotesize
%\bibliographystyle{apalike}
%\bibliography{../biblio}
%\end{frame}

\appendix









\end{document}
