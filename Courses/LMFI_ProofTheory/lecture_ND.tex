\documentclass[xcolor=dvipsnames,aspectratio=169,handout]{beamer}
\usecolortheme{orchid}
\setbeamertemplate{footline}[frame number]


\usepackage[utf8]{inputenc}
\usepackage[english]{babel}
\usepackage[OT1]{fontenc}
\usepackage{hyperref}
\usepackage[inference]{semantic}
\usepackage{graphicx}
\usepackage{amsfonts,amsmath}
\usepackage{cleveref}
\usepackage{scalerel}
\usepackage{multicol}
\usepackage{xspace}
\usepackage{amsmath}
\usepackage{amssymb}
\usepackage{stmaryrd}
\usepackage{proof}
\usepackage{natbib}
\usepackage{booktabs}

\usepackage{mathtools}
\usepackage{mathpartir}
\usepackage{colonequals}
\usepackage{xspace}
% Commands for correct abbreviations


\usepackage{graphicx}
\usepackage{bussproofs}
\usepackage{tikz}
\usetikzlibrary{shapes,arrows,chains}

\EnableBpAbbreviations

\usepackage{tikz}
\usepackage{tikz-cd}
\usetikzlibrary{calc}
\usetikzlibrary{arrows.meta}
\def\checkmark{\tikz\fill[scale=0.4](0,.35) -- (.25,0) -- (1,.7) -- (.25,.15) -- cycle;}

\usepackage{dsfont}

\title[Natural Deduction]{Natural Deduction}
\author[Dominik Kirst]{Dominik Kirst}

\date[October 7th, 2025]{LMFI Proof Theory Course\\October 7th, 2025}


% BOXES
\usepackage[many]{tcolorbox}
\newtcolorbox{emptybox}[1][]{
	beamer,
	width=(0.7\textwidth),
	% enlarge left by=-3pt,
	titlerule=0mm,
	colframe=white,
	coltitle=black,
	bottom=6pt, 
	top=-12pt,
	left=6pt,  
	right=6pt,
	notitle, 
	adjusted title={},
	outer arc=.5mm,
	arc=.5mm,
	no shadow,
	fuzzy shadow={1mm}{-1mm}{-1.2mm}{.7mm}{black!20},
	interior titled code={}
}

\newtcolorbox{widerbox}[1][]{
	beamer,
	width=(0.8\textwidth),
	% enlarge left by=-3pt,
	titlerule=0mm,
	colframe=white,
	coltitle=black,
	bottom=6pt, 
	top=-12pt,
	left=6pt,  
	right=6pt,
	notitle, 
	adjusted title={},
	outer arc=.5mm,
	arc=.5mm,
	no shadow,
	fuzzy shadow={1mm}{-1mm}{-1.2mm}{.7mm}{black!20},
	interior titled code={}
}

\newtcolorbox{widebox}[1][]{
	beamer,
	width=(\textwidth),
	% enlarge left by=-3pt,
	titlerule=0mm,
	colframe=white,
	coltitle=black,
	bottom=6pt, 
	top=-12pt,
	left=6pt,  
	right=6pt,
	notitle, 
	adjusted title={},
	outer arc=.5mm,
	arc=.5mm,
	no shadow,
	fuzzy shadow={1mm}{-1mm}{-1.2mm}{.7mm}{black!20},
	interior titled code={}
}


% TIMELINE
\usepackage{tikz}
\usepackage{graphicx}
\usepackage{stackengine}
\usetikzlibrary{trees, shapes, calc, positioning, shadows}

\definecolor{darkcerulean}{rgb}{0.063661, 0.257392, 0.477463}
\definecolor{scooter}{rgb}{0.161162, 0.775760, 0.885416}


% #1 – box name
% #2 – left color
% #3 – right color
% #4 – Title (may contain #1)
\newcommand{\newlargebox}[4]{
	\newtcolorbox{#1}[1][]{
		beamer,
		width=\textwidth+7pt,
		enlarge left by=-3pt,
		titlerule=3mm,
		colframe=white,
		coltitle={#2},
		bottom=2pt, 
		top=-4pt,
		left=6pt,  
		toptitle=2pt,
		bottomtitle=-2pt,
		fonttitle=\bfseries\large,
		adjusted title={#4},
		outer arc=.5mm,
		arc=.5mm,
		no shadow,
		fuzzy shadow={1mm}{-1mm}{-1.2mm}{.7mm}{black!20},
		interior titled code={
			\path [left color = {#2}, right color = {#3}]
			(title.south west) + (8pt, 0) rectangle ++(\textwidth-1pt, 0.02);
		}
	}
}


\newlargebox{defaultbox}{darkcerulean}{scooter}{#1} 

\usepackage{fancyvrb}
\usepackage{xcolor}


\begin{document}

%\beamerdefaultoverlayspecification{<+->}
\definecolor{red}{RGB}{204,0,0}
\definecolor{yellow}{RGB}{228,242,31}
\definecolor{green}{RGB}{0,204,0}

\newcommand\refs[1]{%
	\begin{textblock*}{8cm}(0.3cm,9.2cm)%
		\scriptsize {\color{gray}#1}
	\end{textblock*}
}

\renewcommand{\to}{\Rightarrow}

\begin{frame}
	\maketitle
\end{frame}

\begin{frame}{Slogans}
	\begin{itemize}
		\item
		Deduction system closely resembling natural reasoning
		\vspace{0.5cm}
		\item
		Focus on \emph{introduction} and \emph{elimination} rules
		\vspace{0.5cm}
		\item
		Intuitionistic and classical (and minimal) variants
		\vspace{0.5cm}
		\item
		Interesting completeness proofs
		\vspace{0.5cm}
		\item
		Connection to algebraic / computational / categorical semantics
		\vspace{0.5cm}
		\item
		Originally developed by Gentzen in the context of consistency proofs
	\end{itemize}
\end{frame}

\begin{frame}{Overview}
	Lecture 1:
	\begin{itemize}
		\item
		Intuitionistic and classical natural deduction
		\item
		Structural properties
		\item
		Equivalence to sequent calculus
		\item
		Double-negation translation
	\end{itemize}
	
	\vspace{0.2cm}
	Lecture 2:
	\begin{itemize}
		\item
		Henkin's completeness proof
		\item
		Kripke semantics
		\item
		Semantic cut-elimination
	\end{itemize}
	
	\vspace{0.2cm}
	Lecture 3:
	\begin{itemize}
		\item
		Algebraic semantics
		\item
		Friedman translation
	\end{itemize}
\end{frame}

\begin{frame}{Intuitionistic Natural Deduction (NJ)}
	\begin{gather*}
		\inference[Ax]{A\in \Gamma}{\Gamma\vdash A}\hspace{2em}
		\inference[Exp]{\Gamma\vdash\bot}{\Gamma\vdash A}\hspace{2em}
		\inference[II]{\Gamma,A\vdash B}{\Gamma\vdash A\to B}\hspace{2em}
		\inference[IE]{\Gamma\vdash A\to B& \Gamma \vdash A}{\Gamma \vdash B}\\[0,5cm]
		\inference[CI]{\Gamma\vdash A&\Gamma\vdash B}{\Gamma \vdash A\land B}\hspace{2em}
		\inference[CE1]{\Gamma\vdash A\land B}{\Gamma \vdash A}\hspace{2em}
		\inference[CE2]{\Gamma\vdash A\land B}{\Gamma \vdash B}\\[0,5cm]
		\inference[DI1]{\Gamma\vdash A}{\Gamma\vdash A\lor B}\hspace{2em}
		\inference[DI2]{\Gamma\vdash B}{\Gamma\vdash A\lor B}\hspace{2em}
		\inference[OE]{\Gamma\vdash A\lor B&\Gamma,A\vdash C&\Gamma,B\vdash C}{\Gamma \vdash C}\\[0,5cm]
		\inference[AI]{\Gamma \vdash A & x\not\in FV(\Gamma)}{\Gamma\vdash \forall x A}\hspace{2em}
		\inference[AE]{\Gamma\vdash \forall x A}{\Gamma\vdash A[t/x]}\\[0,5cm]
		\inference[EI]{\Gamma\vdash A[t/x]}{\Gamma\vdash \exists x A}\hspace{2em}
		\inference[EE]{\Gamma\vdash \exists x A&\Gamma,A\vdash B&x\not\in FV(\Gamma,B)}{\Gamma\vdash B}
	\end{gather*}
\end{frame}

\begin{frame}{Classical Natural Deduction (NK)}
	To obtain a classical system $\Gamma\vdash_c A$ replace the Exp rule by
	$$\inference[DN]{\Gamma,\neg A\vdash_c\bot}{\Gamma\vdash_c A}$$
	where here we treat $\neg A$ as a shorthand for $A\to \bot$.
	
	\pause
	
	\vspace{0.5cm}
	We will establish a few expected properties:
	\begin{itemize}
		\item
		If $\Gamma\vdash A$ then $\Gamma\vdash_c A$.
		\item
		We always have $\Gamma\vdash_c A\lor\neg A$.
		\item
		We don't always have $\Gamma\vdash A\lor\neg A$.
	\end{itemize}
	
	\pause
	
	\vspace{0.5cm}
	There is also a minimal system $\Gamma \vdash_m A$ dropping Exp without replacement:
	\begin{itemize}
		\item
		Obviously, if $\Gamma\vdash_m A$ then $\Gamma\vdash A$ and $\Gamma\vdash_c A$.
	\end{itemize}
\end{frame}

\begin{frame}{Example Derivation: Excluded Middle}
	\pause
	$$
	\infer{\vdash_c A\lor\neg A}{
	\infer{\neg( A\lor \neg A)\vdash\bot}{
	\infer{\neg( A\lor \neg A)\vdash (A\lor \neg A)\to \bot}{} &
	\infer{\neg( A\lor \neg A)\vdash A\lor \neg A}{
	\infer{\neg( A\lor \neg A)\vdash \neg A}{
	\infer{\neg( A\lor \neg A),A\vdash \bot}{
	\infer{\neg( A\lor \neg A),A\vdash (A\lor \neg A)\to \bot}{} &
	\infer{\neg( A\lor \neg A),A\vdash A\lor\neg A}{
	\infer{\neg( A\lor \neg A),A\vdash A}{}
	}}}}}}
	$$
	
	\pause
	\vspace{0.5cm}
	Note that already $\neg( A\lor \neg A)\vdash_m \bot$ holds!
\end{frame}

\begin{frame}{Example Derivation: Dual Drinker Paradox}
	$$
	\scriptsize
	\infer{\vdash_c \exists x(\exists yA(y)\to A(x))}{
	\infer{\vdash_c \exists z A(z)\lor\neg \exists z A(z)}{\text{previous slide}} &
	\infer{\exists z A(z) \vdash_c DD}{
	\infer{\exists z A(z) \vdash_c\exists z A(z) }{} &
	\infer{\exists z A(z), A(z)\vdash_c DD}{
	\infer{\exists z A(z), A(z)\vdash_c \exists yA(y)\to A(z)}{
	\infer{\exists z A(z), A(z), \exists yA(y) \vdash_c A(z)}{}
	}
	}
	}
	&
	\infer{\neg\exists z A(z) \vdash_c DD}{
	\infer{\neg\exists z A(z)\vdash_c \exists yA(y)\to A(u)}{
	\infer{\neg\exists z A(z),\exists z A(z)\vdash_c A(u)}{
	\infer{\neg\exists z A(z),\exists z A(z),\neg A(u)\vdash_c \bot}{
	\infer{\Gamma \vdash_c \exists z A(z) \to \bot}{} & \infer{\Gamma \vdash_c\exists z A(z) }{}
	}
	}
	}
	}
	}
	$$
	
	\vspace{0.5cm}
	Here $u$ is some term, $\Gamma$ is $\neg\exists z A(z),\exists z A(z),\neg A(u)$ and $DD$ is $\exists x(\exists yA(y)\to A(x))$.
\end{frame}

\begin{frame}{Structural Properties}
	Recall the structural properties we had for sequent calculus:
	\begin{itemize}
		\item
		Contraction: if $\Gamma,A,A\vdash \Delta$ then $\Gamma,A\vdash \Delta$
		\item
		Exchange: if $\Gamma,A,B\vdash\Delta$ then $\Gamma,B,A\vdash\Delta$
		\item
		Weakening: if $\Gamma\vdash\Delta$ then $\Gamma,A\vdash \Delta$
	\end{itemize}
	
	\pause
	
	\vspace{0.5cm}
	In our formulation of ND based on sets, they all follow from weakening:
	\vspace{0.5cm}
	\begin{lemma}
		If $\Gamma\vdash A$ and $\Gamma\subseteq \Delta$, then $\Delta\vdash A$.
	\end{lemma}
	\pause
	\begin{proof}[Proof strategy]
		Straightforward induction on $\Gamma\vdash A$ for the propositional rules but surprisingly tricky for the quantifier rules due to freshness conditions.
	\end{proof}
\end{frame}

\begin{frame}{Direct Consequences of Weakening}
	\begin{itemize}
		\pause
		\item
		The Exp rule is admissible in NK, so NK subsumes NJ:
		$$
		\infer{\Gamma\vdash_c A}{
		\infer{\Gamma,\neg A \vdash_c \bot}{\Gamma \vdash_c \bot}
		}
		$$
		\vspace{0.3cm}
		\pause
		\item
		The II rule is invertible:
		$$
		\infer{\Gamma,A\vdash B}{
		\infer{\Gamma,A\vdash A\to B}{\Gamma\vdash A\to B}
		&
		\infer{\Gamma,A\vdash A}{}
		}
		$$
	\end{itemize}
\end{frame}

\begin{frame}{Equivalence of LJ and NJ}
	\begin{theorem}
		If $\Gamma\vdash \Xi$ is derivable in LJ, then $\Gamma\vdash \bigvee \Xi$ is derivable in NJ.
	\end{theorem}
	\pause
	\begin{proof}[Proof strategy]
		By induction on the derivation of $\Gamma\vdash \Xi$, boiling down to showing all rules of LJ admissible in NJ.
	\end{proof}
	
	\pause
	\vspace{0.5cm}
	Since we already argued the converse, we obtain:
	\begin{itemize}
		\item
		NJ is consistent
		\item
		Propositional NJ is decidable
		\item
		NJ has the disjunction and witness properties
		\item
		Completeness results for LJ apply to NJ
		%\item
		%More subtle: NJ has cut-elimination
	\end{itemize}
\end{frame}

\begin{frame}{Equivalence to LJ and NJ: Interesting Cases}
	The non-trivial rules to show admissible are:
	
	\begin{gather*}
		\inference[cut]{\Gamma_1\vdash A & \Gamma_2,A\vdash B}{\Gamma_1,\Gamma_2\vdash B}\hspace{2em}
		\inference[RW]{\Gamma\vdash\bot}{\Gamma\vdash A}\\[0,5cm]
		\inference[L$\neg$]{\Gamma \vdash A}{\Gamma,\neg A\vdash \bot}\hspace{2em}
		\inference[L$\to$]{\Gamma_1\vdash A & \Gamma_2,B\vdash C}{\Gamma_1,\Gamma_2,A\to B\vdash C}\\[0,5cm]
		\inference[L$\land$1]{\Gamma,A\vdash C}{\Gamma,A\land B\vdash C}\hspace{2em}
		\inference[L$\land$2]{\Gamma,B\vdash C}{\Gamma,A\land B\vdash C}\hspace{2em}
		\inference[L$\lor$]{\Gamma,A\vdash C & \Gamma,B\vdash C}{\Gamma, A\lor B\vdash C}\\[0,5cm]
		\inference[L$\forall$]{\Gamma,A[t/x]\vdash B}{\Gamma,\forall x A\vdash B}\hspace{2em}
		\inference[L$\exists$]{\Gamma, A\vdash C & x\not\in FV(\Gamma,C)}{\Gamma,\exists x A\vdash C}
	\end{gather*}
\end{frame}

\begin{frame}{Equivalence of LK and NK}
	\begin{theorem}
		If $\Gamma\vdash \Delta$ is derivable in LK, then $\Gamma\vdash \bigvee \Delta$ is derivable in NK.
	\end{theorem}
	\pause
	\begin{proof}[Proof strategy]
		By induction on the derivation of $\Gamma\vdash \Delta$, boiling down to showing all rules of LK admissible in NK.
	\end{proof}
	
	\pause
	\vspace{0.5cm}
	Since we already argued the converse, we obtain:
	\begin{itemize}
		\item
		NK is consistent
		\item
		Propositional NK is decidable
		\item
		Completeness results for LK apply to NK
		%\item
		%More subtle: NJ has cut-elimination
	\end{itemize}
\end{frame}

\begin{frame}{Double-Negation Translation}
	In propositional logic, the relation of classical and intuitionistic flavour is simple:
	\begin{theorem}[Glivenko]
		If $A$ is propositional, then $\vdash_c A$ if and only if $\vdash \neg\neg A$.
	\end{theorem}
	
	\pause
	\vspace{0.3cm}
	In first-order logic, the situation is slightly more subtle, for instance define:
	\begin{align*}
		P^* &:= \neg\neg P & (A\lor B)^* &:=\neg\neg (A^*\lor B^*)\\
		(A\to B)^* &:= A^*\to B^* & (\forall x A)^* &:=\forall  x A^*\\
		(A\land B)^* &:=A^*\land B^* & (\exists x A)^* & := \neg\neg \exists x A^*
	\end{align*}
	
	\pause
	\begin{theorem}[Gödel-Gentzen]
		For arbitrary $A$, we have $\vdash_c A$ if and only if $\vdash A^*$.
	\end{theorem}
	\vspace{0.3cm}
	\pause
	Consequences: relative consistency/decidability, extraction of computational content...
\end{frame}

\begin{frame}{Double-Negation Translation: Proof Outline}
	\begin{enumerate}
		\pause
		\item
		Classically, $A$ and $A^*$ are equivalent: $\vdash_c A\Leftrightarrow A^*$
		\begin{itemize}
			\pause
			\item
			Straightforward, exploiting that classical logic identifies $B$ and $\neg\neg B$
		\end{itemize}
		\vspace{0.7cm}
		\pause
		\item
		Double-negation elimination holds for translated formulas: $\vdash \neg\neg A^*\to A^*$
		\begin{itemize}
			\pause
			\item
			By induction on $A$, exploiting that intuitionistic logic identifies $\neg B$ and $\neg\neg\neg B$
		\end{itemize}
		\vspace{0.7cm}
		\pause
		\item
		Derivations in contexts can be replayed in the translation: $\Gamma\vdash_c A$ implies $\Gamma^*\vdash A^*$
		\begin{itemize}
			\pause
			\item
			By induction on $\Gamma\vdash_c A$, using (2) whenever negative reasoning is needed
		\end{itemize}
	\end{enumerate}
\end{frame}

\begin{frame}{Double-Negation Translation: Some Interesting Cases}
	The most interesting cases are disjunction and existential elimination:
	\vspace{0.7cm}
	\begin{gather*}
		\inference{\Gamma^*\vdash \neg\neg (A^*\lor B^*) & \Gamma^*,A^*\vdash C^* & \Gamma^*,B^*\vdash C^*}
		{\Gamma^*\vdash C^*}\\[0.7cm]
		\inference{\Gamma^*\vdash \neg\neg\exists xA^* & \Gamma^*,A^*\vdash B^* & x\notin FV(\Gamma^*,B^*)}
		{\Gamma^*\vdash C^*}
	\end{gather*}
\end{frame}


\end{document}



\begin{frame}{Henkin's Completeness Proof}
	content
\end{frame}

\begin{frame}{Henkin's Completeness Proof: Some Cases}
	content
\end{frame}



%\begin{frame}{Bibliography}
%%\begin{frame}{Bibliography}
%\footnotesize
%\bibliographystyle{apalike}
%\bibliography{../biblio}
%\end{frame}

\appendix









\end{document}
